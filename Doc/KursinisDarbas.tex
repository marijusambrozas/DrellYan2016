%Kompiliuoti su XeLaTeX ir BibTeX

\documentclass[a4paper, 12pt]{article}

\usepackage[yyyymmdd]{datetime}

\usepackage{fontspec}
\usepackage{fontenc}
\usepackage{ulem}
\usepackage{cite}
\usepackage{mathtools}
\usepackage{amsmath}
\usepackage{amssymb}
%\usepackage{float}
\usepackage{graphicx}
\usepackage{multirow}
\usepackage[hyphens]{url}
\usepackage{caption}
\usepackage[svgnames]{xcolor}
\usepackage{lineno}
\usepackage[lithuanian]{babel}
\usepackage{hyperref}
\usepackage{siunitx}
\usepackage{floatrow}

\floatsetup[table]{capposition=top}

\hypersetup{breaklinks=true}
\urlstyle{same}

\usepackage{geometry}
\pagestyle{myheadings}
\geometry{
	left=3cm,
	right=1cm,
	top=2cm,
	bottom=2cm,
}
\pagenumbering{arabic}
\linespread{1.25}

\graphicspath{ {images/} }

\renewcommand{\dateseparator}{-}
\addto\captionslithuanian{\renewcommand{\figurename}{pav}}
\addto\captionslithuanian{\renewcommand{\refname}{Literatūros sąrašas}}
\addto\captionslithuanian{\renewcommand{\tablename}{lentelė}}

\DeclareCaptionLabelFormat{numfirst}{#2~#1}
\captionsetup[figure]{labelformat = numfirst, labelsep = period}
\captionsetup[table]{labelformat = numfirst, labelsep = period}

\newcommand{\textblue}[1]{{\color{Blue}#1}}
\newcommand{\textred}[1]{{\color{Red}#1}}
\newcommand{\comment}[1]{\newline\textblue{#1}\newline}
\newcommand{\commentNL}[1]{\textblue{#1}\newline}
\newcommand{\commentMA}[1]{\textred{#1}\newline}
\newcommand{\ttt}[1]{\texttt{#1}}
\newcommand{\pT}{p_{\mathrm{T}}}
\newcommand{\ET}{E_{\mathrm{T}}}
\newcommand{\WW}{W\! W}
\newcommand{\ZZ}{Z\! Z}
\newcommand{\WZ}{W\! Z}
\newcommand{\tbarW}{\bar{t}W}
\newcommand{\ttbar}{t\bar{t}}
\newcommand{\emu}{e\mu}
\newcommand{\mumu}{\mu\mu}
\newcommand{\gJets}{\gamma\! +\!\mathrm{Jets}}
\newcommand{\WJets}{W\! +\!\mathrm{Jets}}
\newcommand{\dtW}{tW\! + \! \bar{t}W}
\newcommand{\DYee}{\mathrm{DY} \! \rightarrow \! ee}
\newcommand{\DYtau}{\mathrm{DY} \! \rightarrow \! \tau\tau}
\newcommand{\DY}{\mathrm{DY}}
\newcommand{\ltq}[1]{{\quotedblbase{}#1\textquotedblleft{}}}
\newcommand{\Lumi}{{\cal L}_\mathrm{int}}
\newcommand{\invfb}{fb$^{-1}$}
\newcommand{\invpb}{pb$^{-1}$}
\newcommand{\QCD}{QC\! D}

\newlength\q
\setlength\q{\dimexpr .5\textwidth -2\tabcolsep}

\begin{document}
\linenumbers

\begin{titlepage}
\centering
{\large Vilniaus universitetas \\ Fizikos fakultetas \\ Teorinės fizikos ir astronomijos institutas \par}
\vspace{3.5cm}
{\Large Marijus Ambrozas \par}
\vspace{0.3cm}
{\LARGE Drell-Yan proceso tyrimas naudojant 2016 metų CERN CMS duomenis \par}
\vspace{0.8cm}
{\large Mokslinis tiriamasis darbas \par}
\vspace{0.8cm}
{\large Teorinės fizikos ir astrofizikos studijų programa \par}
\vspace{3.5cm}
{\large \begin{tabular*}{0.9\textwidth}{@{\extracolsep{\fill}}ll}
Studentas & Marijus Ambrozas\tabularnewline[0.5cm]
Darbo vadovas & dr. Andrius Juodagalvis\tabularnewline[0.5cm]
\end{tabular*} \par}
\vspace{4cm}
{\large Vilnius $2018$\par}
\end{titlepage}


\clearpage
\addtocounter{page}{1}
\addtocontents{toc}{\protect\setcounter{tocdepth}{2}}
\tableofcontents
\clearpage

\section*{Įvadas} \addcontentsline{toc}{section}{Įvadas}
Dalelių susidūrimai, vykdomi CERN Didžiajame hadronų greitintuve (angl.\
\textit{Large Hadron Collider} -- LHC), suteikia galimybę vis giliau pažvelgti į mažiausius
Visatos statybinius blokus bei tarp jų vykstančias sąveikas, taip pat ieškoti atsakymų
į dar neatsakytus klausimus ir tikrinti naujas teorijas.
Kad būtų galima surinkti kuo daugiau informacijos apie labai retai įvykstančius
dalelių sąveikos procesus, greitintuve yra didinamas registruojamų protonų susidūrimų
skaičius: pavyzdžiui, per 2016 metus buvo užregistruota maždaug 10 kartų daugiau protonų
susidūrimų, nei per 2015 metus.
Tai sudaro nemažą iššūkį mokslininkams, kurie turi nuspręsti, kur reikia saugoti
didžiulius kiekius duomenų, taip pat, kaip sumažinti jų analizės trukmę.

Norint kuo geriau suprasti tai, kas buvo užregistruota protonų susidūrimus
fiksuojančiuose detektoriuose, labai svarbu turėti kaip įmanoma tikslesnį protonų
sandaros bei jų tarpusavio sąveikos aprašymą.
Tokiu būdu galima lyginti teoriškai numatomus rezultatus su užregistruotaisiais eksperimento
metu.
Teorinėje dalelių fizikoje protonų sandara aprašoma partonų pasiskirstymo funkcijomis
(angl.\ \textit{parton distribution functions} -- PDF), nuo kurių tikslumo kone labiausiai
priklauso teorinių įverčių kokybė.
Teorinių modelių tobulinimui nemažą svarbą turi eksperimentinis Drell-Yan proceso tyrimas.
Tikslūs Drell-Yan diferencialinio sklaidos skerspjūvio matavimai leidžia ne tik tik tikslinti
partonų pasiskirstymo funkcijas bei teorinius modelius, bet taip pat palengvina ir kitus
eksperimentinės didelių energijų fizikos tyrimus, kur Drell-Yan procesas yra dominuojantis triukšmas.

Vis dėlto, didelis duomenų kiekis nėra vienintelis sunkumas šio proceso tyrime -- eksperimentiniuose
duomenyse kartais pasitaiko užfiksuotų triukšmo įvykių, kurių indėlį į gaunamą rezultatą reikia
įvertinti.
Neretai įvertinti triukšmo įvykių skaičiui vien kompiuterinio modeliavimo nepakanka.
Tokiais atvejais į pagalbą pasitelkiami matavimu grįsti metodai.
Tam tikrų Drell-Yan triukšmo procesų įvykių skaičių galima įvertinti $\emu$ metodu.
Šio darbo tikslas -- iš didelės apimties pirminio duomenų rinkinio atrinkti su Drell-Yan
procesu siejamus protonų susidūrimo įvykius, bei $\emu$ metodu įvertinti, kokią dalį atrinktų
duomenų rinkinyje užima triukšmo įvykiai. Šiam tikslui įgyvendinti buvo naudojami 2016 metais
CERN CMS detektoriaus užregistruoti $13$ TeV energijos protonų susidūrimų duomenys.

\clearpage

\section{Drell-Yan procesas ir jo tyrimas}


\subsection{Protono sandara ir Drell-Yan procesas}


\subsubsection{Partonų pasiskirstymo funkcijos}
Teorinėje elementariųjų dalelių fizikoje, protonų, kaip ir kitų hadronų, sandara aprašoma
partonų pasiskirstymo funkcijomis. Tikslus šių funkcijų žinojimas yra ypatingai svarbus
norint padaryti bet kokius teorinius hadronų greitintuvuose vykstančių įvykių įverčius.
Partonų pasiskirstymo funkcija $f_{i}(x, Q^{2})$ aprašo tikimybę aptikti protono impulso
dalį $x$ nešantį $i$ tipo partoną (pavyzdžiui, kylantyjį kvarką -- $u$, krentantyjį
kvarką -- $d$ ir t.t.), kai \textit{kietojo} susidūrimo (angl. \textit{hard interaction} --
procesas, kurio metu smarkiai pasikeičia dalelių impulsas) energiją aprašantis dydis lygus $Q$.
Kai kokia nors dalelė yra sklaidoma hadrono, teoriniai susidūrimo metu galimų reakcijų
skerspjūviai yra apskaičiuojami kaip partonų pasiskirstymo funkcijos ir sklaidomos dalelės
sąveikos su tam tikru partonu skerspjūvio kombinacija.
Partonų pasiskirstymo funkcijos negali būti apskaičiuotos pasinaudojant kvantinės chromodinamikos
(angl. \textit{quantum chromodynamics} -- QCD) teorija, todėl jos yra apskaičiuojamos
pasinaudojant įvairių procesų eksperimentinių tyrimų rezultatais \cite{Placakyte:2011az}.

Partonų pasiskirstymo funkcijų tikslinimui jau ilgą laiką pasitarnauja Drell-Yan proceso eksperimentinis
tyrimas. 

\vspace{0.2cm}
\begin{centering}
\begin{minipage}[t]{0.48\linewidth}
\centering
\includegraphics[width=0.7\linewidth]{NNPDF10.PNG}
\end{minipage}
\hfill
\begin{minipage}[t]{0.48\linewidth}
\centering
\includegraphics[width=0.7\linewidth]{NNPDF10000.PNG}
\end{minipage}
\vspace{-0.3cm}
\captionof{figure}{ \label{fig:PDFs}
NNPDF partonų pasiskirstymo funkcijų pavyzdžiai \cite{NNPDF}. Kairėje pusėje matomos protono partonų pasiskirstymo funkcijos, kai kvadratinė energija $\mu^{2}=10 \; \mathrm{GeV}^{2}$ o dešinėje -- kai $\mu^{2}=10000 \; \mathrm{GeV}^{2}$. Ant abscisių ašių vaizduojama partono turima protono impulso dalis $x$, o ant ordinačių ašių -- tikimybės tankio vertė $f(x,\; \mu)$, padauginta iš $x$.
}
\end{centering}


\subsubsection{Drell-Yan procesas}
Drell-Yan procesas -- tai toks procesas, kai anihiliavus kvarkui ir antikvarkui (protonų susidūrimo
metu) sukuriama leptono-antileptono pora.
Šis procesas vyksta apsikeičiant $Z$ bozonu arba
virtualiu fotonu per s-kanalą:

\begin{equation*}
	q\bar{q} \rightarrow Z/ \gamma^{*} \rightarrow l^{+}l^{-}
\end{equation*}

Toliau bus žymima $\DY \! \rightarrow \! l^{+}l^{-}$. Drell-Yan proceso Feinmano diagrama
pavaizduota \ref{fig:DYfeyn} paveiksle.

Drell-Yan procesas tapo svarbiu tyrimo objektu nuo pat pirmojo jo aprašymo $1970$-aisiais
metais, kuriuo S.D.\ Drell ir T.M.\ Yan bandė apibūdinti leptonų-antileptonų porų susidarymą
hadronų susidūrimų metu \cite{DYoriginal}.

Šiais laikais Drell-Yan procesas naudojantis pertubacine kvantine chromodinamika teoretikų
gali būti aprašomas trečios eilės perturbacijų tikslumu (angl. \textit{next-to-next-to-leading
order} -- NNLO).
Tikslūs eksperimentiniai Drell-Yan proceso diferencialinio skerspjūvio matavimai naudojami
ne tik partonų pasiskirstymo funkcijoms tikslinti, bet taip pat ir perturbacinės kvantinės
chromodinamikos, elektrosilpnosios sąveikos teorijoms tikrinti. Taip pat šis procesas yra vienu iš
pagrindinių triukšmo procesų įvairiuose kituose tyrimuose, todėl tikslūs Drell-Yan proceso matavimai
įvairiapusiškai prisideda prie kitų didelių energijų fizikos tyrimų rezultatų kokybės.

\begin{figure}[H]
\centering
\includegraphics[scale=0.75]{DYprocess.PNG}
\caption{Drell-Yan proceso Feynman'o diagramos.}
\label{fig:DYfeyn}
\end{figure}


\subsubsection{Skerspjūvis}

Drell-Yan proceso, taip pat, kaip ir kitų didelių energijų fizikos procesų tikėtinumo aprašymui
naudojamas skerspjūvis, kuris dažniausiai matuojamas barnais ($1$ b $= 10^{-28}$ m$^{2}$).
Pagrindinis skerspjūvio pranašumas prieš bedimensinę tam tikro įvykio tikimybę yra tas, kad
pastarosios vertė priklauso nuo greitintuvuose lekiančių dalelių spindulių matmenų bei tankių.
Tuo tarpu skerspjūvio vertė išlieka pastovi, tad galima nesunkiai lyginti tam tikrų įvykių
skerspjūvius, išmatuotus skirtinguose dalelių greitintuvuose.

Norint labiau pasigilinti į tam tikras tiriamo proceso charakteristikas būna naudojami diferencialiniai
skerspjūviai $\mathrm{d}\sigma/\mathrm{d}\xi$: funkcijos, apibūdinančios, tikėtinumą, kad mus
dominantis įvykis ne tik įvyks, bet dar ir mūsų pasirinktas rezultatą apibūdinantis tolydus
dydis $\xi$ pateks į tam tikrą intervalą $\mathrm{d}\xi$.

Drell-Yan proceso tyrime dažnai matuojamas diferencialinis skerspjūvis, priklausantis nuo
leptonų poros invariantinės masės, skersinio impulso, arba spartos:
$\mathrm{d}\sigma / \mathrm{d}m$, $\mathrm{d}\sigma / \mathrm{d}p_{\mathrm{T}}$, arba
$\mathrm{d}\sigma / \mathrm{d}y$.
Taip pat galimi ir dvimačio arba trimačio diferencialinio skerspjūvio matavimai,
priklausantys nuo kelių (arba visų) iš išvardintų dydžių.
Ganėtinai išsiskiriančią charakteristiką turi nuo leptonų poros invariantinės masės priklausantis
diferencialinis skerspjūvis, dar vadinamas invariantinės masės spektru -- jis yra rezonansinė
kreivė, t.y. turi piką ties $Z$ bozono mase (apie $91.2$ GeV).
Nuo masės priklausančio diferencialinio skerspjūvio pavyzdys pateikiamas \ref{fig:DYeeCS} paveiksle.


\subsubsection*{Invariantinė masė} 

Invariantinė masė -- tai masė, kuri nepriklauso nuo atskaitos sistemos.
Skaičiuojant vienos dalelės invariantinę masę, ji sutampa su jos rimties mase:
\begin{equation}
	m_{0}^{2} = E^{2} - | \vec{p}\, |^{2} \; .
	\label{eq:invm}
\end{equation}
čia $m_{0}$ -- dalelės invariantinė masė, $E$ -- energija, $\vec{p}$ -- impulso vektorius.
Taip pat šioje formulėje naudojama vienetų sistema, kurioje šviesos greitis prilygintas vienetui.
Tokioje sistemoje dalelės masė, impulsas ir energija matuojama tais pačiais vienetais, dažniausiu atveju
gigaelektronvoltais (GeV).

Taip pat galima apskaičiuoti ir kelių dalelių sistemos invariantinę masę.
Pastaroji nėra lygi dviejų dalelių rimties masių sumai.
Jeigu dalelės, kurių sistemai skaičiuojame invariantinę masę, yra kažkokios kitos dalelės
skilimo produktai, tai gauta vertė bus lygi motininės dalelės rimties masei.
Kelių dalelių sistemos invariantinė masė skaičiuojama taip:
\begin{equation}
	m_{\mathrm{sistemos}}^{2} = \left( \sum_{n=1}^{N} E_{n} \right)^{2} -
	\left| \sum_{m=1}^{N} \vec{p}_{m} \right|^{2} \; .
	\label{eq:minvm}
\end{equation}
Dviejų dalelių atveju:
\begin{equation}
	m_{12}^2 = ( E_{1} + E_{2} )^{2} - | \vec{p}_{1} + \vec{p}_{2} |^{2} \; .
	\label{eq:tinvm}
\end{equation}
Būtent taip yra skaičiuojama Drell-Yan proceso metu susidariusių dviejų leptonų invariantinė masė.

\begin{centering}
\begin{figure}[H]
\centering
\includegraphics[scale=0.6]{DYeeCS.PNG}
\vspace{-0.2cm}
\caption{\label{fig:DYeeCS}
Drell-Yan proceso diferencialinis sklaidos skerspjūvis, priklausantis nuo leptonų poros invariantinės
masės.
Protonų susidūrimo energija -- $8$ TeV.
Juodi taškai žymi eksperimentinio matavimo rezultatus, o mėlyna ištisinė linija -- teoriškai numatytą
rezultatą \cite{DYpic}.
% GALIMA ĮDĖTI ŠITĄ https://cds.cern.ch/record/2648776/files/SMP-17-001-paper-v14.pdf
% 13 TEV, BET ČIA DRAFT VERSIJA
}
\end{figure}
\end{centering}


\subsubsection*{Sparta ir pseudosparta}

Sparta (angl.\ \textit{rapidity}) -- tai gan dažnai dalelių fizikoje naudojamas dydis, kuris yra patogus tuo,
kad skirtumas tarp dviejų dalelių spartų nepakinta perėjus į bet kokią kitą išilgai cilindrinių koordinačių
$z$ ašies judančią sistemą -- yra invariantas.
Sparta yra žymima raide $y$ ir apibrėžiama tokia formule:

\begin{equation}
	y = \frac{1}{2} \ln{ \left( \frac{E+p_{z}c}{E-p_{z}c} \right) } \; \mathrm{,}
	\label{eq:rapidity}
\end{equation}

čia $E$ -- dalelės energija, $p_{z}$ -- dalelės impulso projekciją į $z$ ašį, $c$ -- šviesos greitis.

Kai dalelė juda reliatyvistiniais greičiais ir jos rimties energija, lyginant su pilnutine energija yra labai
mažas dydis, tai pilnutinę energiją galima aproksimuoti kinetine energija:

\begin{equation}
	E^2 = m^2c^4 + |\vec{p}|^2c^2 \approx |\vec{p}|^2c^2
	\label{eq:relEnergy}
\end{equation}

Galima parodyti, kad tokiu atveju spartos apibrėžimas supaprastėja iki tokios išraiškos:

\begin{equation}
	y_{v\rightarrow c} = -\ln \left( \tan \frac{\theta}{2} \right) = \eta \; \mathrm{,}
	\label{eq:pseudorapidity}
\end{equation}

Tokia spartos aproksimacija vadinama pseudosparta ir žymima raide $\eta$.
Taip pat čia $\theta$ -- kampas, kurį sudaro dalelės impulso vektorius su $z$ ašimi.
Šiuolaikiniuose dalelių greitintuvuose, kai dalelės juda beveik šviesos greičiu ir kai tiksliai
išmatuoti dalelės impulsą arba energiją išilgai $z$ ašies yra neįmanoma, vietoje spartos
yra naudojama pseudosparta.
Tiek sparta, tiek pseudosparta esant $\theta=\ang{90}$ yra lygios nuliui, o esant $\theta=\ang{0}$ --
begalybei (arba minus begalybei, kai $\theta=\ang{180}$).


\subsubsection*{Skersinis impulsas ir skersinė energija}
Kaip kątik buvo paminėta, dalelių detektoriui esant cilindriškai išdėstytam aplink protonų spindulį,
dalelės impulso projekcijos į $z$ ašį nustatyti yra neįmanoma.
Todėl šiuolaikiniuose dalelių detektoriuiose, tokiuose kaip CERN CMS yra matuojama tik statmena $z$
ašiai dalelės impulso dedamoji arba dalelės judėjimo statmenai $z$ ašiai nulemtos kinetinės energijos
dedamoji.
Šie dydžiai yra vadinami skersiniu impulsu $\pT$ (angl.\ \textit{transverse momentum}) ir
skersine energija $\ET$ (angl.\ \textit{transverse energy}).
Išmatavus skersinį impulsą kartu su dalelės sparta $\eta$ ir azimutiniu kampu $\phi$ bei žinant dalelės
masę galima pilnai aprašyti dalelės judėjimą taip pat, kaip ir žinant dalelės pilnutinę energiją bei
impulso dekartines dedamąsias.

Taip pat galima pastebėti, kad žinant dalelės skersinį impulsą ir pseudospartą, galima apskaičiuoti ir
jos pilnutinio impulso vertę:

\begin{equation}
	|\vec p|=p_{\mathrm{T}}\cosh\eta \; .
	\label{eq:pmodpt}
\end{equation}


\subsubsection*{Šviesis}

Šviesis -- tai dydis, apibūdinantis dalelių, per laiko vienetą pralėkusių mus
dominantį plotą, skaičius. Šviesis žymimas raide $\mathcal{L}$ (angl. \textit{Luminosity}).

Norint apibūdinti dalelių greitintuvuose vykstančių susidūrimų skaičių, naudojamas šviesio
laikinis integralas -- integruotas šviesis.

\begin{equation}
	\Lumi=\int_{t_1}^{t_2} \mathcal{L}\mathrm{d}t =
	\int_{t_1}^{t_2} \frac{1}{\sigma}\frac{\mathrm{d}N}{\mathrm{d}t} \mathrm{d}t \; .
\label{eq:lumiInt}
\end{equation}

Iš \eqref{eq:lumiInt} išraiškos galima matyti, kad integruotas šviesis turi atvirkštinio ploto
dimensiją.
Didelių energijų fizikos eksperimentatoriai šį dydį matuoja atvirkštiniais barnais -- $b^{-1}$.
Taip pat galima pastebėti, kad sudauginus integruotą šviesį su tam tikro įvykio skerspjūviu
galime gauti, kiek kartų tas įvykis turėtų įvykti -- taip pat, kaip ir sudauginus įvykio tikimybę
su bandymų skaičiumi:

\begin{equation}
	N=\sigma\Lumi \; .
\label{eq:evno}
\end{equation}

Integruotas šviesis, lyginant su bandymų skaičiumi turi tokius pat privalumus, kaip skerspjūvis,
lyginant su tikimybe.
Eksperimentinėje didelių energijų fizikoje labai svarbu tiksliai apskaičiuoti integruotą šviesį,
nes jis susieja teoriją (įvykio skerspjūvi) su tuo, kas fiksuojama eksperimente (įvykių skaičius).


\subsection{Drell-Yan proceso eksperimentinis tyrimas}

\subsubsection{Signalas ir triukšmas}\label{sec:SignalBkg}

CERN Didžiajame hadronų greitintuve vykstančių $13$ TeV energijos protonų susidūrimų metu gali
būti sukurtos labai masyvios ir labai trumpai gyvuojančios dalelės.
Tokios dalelės skyla nespėjusios pasiekti detektoriaus, todėl fiksuojami yra tik tokių dalelių
skilimo produktai -- ilgiau gyvuojančios ir mums gerai pažįstamos dalelės (pavyzdžiui,
elektronai, fotonai, įvairių rūšių hadronai, miuonai).
Detektoriuje užfiksuojamos dalelės su savo parametrais yra vadinamos įvykio galutine būsena
(angl. \textit{final state}).
Ko gero akivaizdu, kad vykdant tokį eksperimentą yra sunku vienareikšmiškai spręsti, kad
detektoriuje užfiksuotos dalelės yra būtent tam tikros dalelės skilimo produktai.
Pavyzdžiui, ieškant Higso bozono galima ieškoti įvykio, kurio metu Higso bozonas skilo į du
$Z$ bozonus, kurie abu skilo į leptonus: $H \rightarrow ZZ \rightarrow l^{+}l^{-}l^{+}l^{-}$.
Tačiau pamatę detektoriuje užfiksuotus keturis leptonus negalime teigti, kad užfiksavome
Higso bozono įvykį.
Daug dažniau protonų susidūrimo metu gali būti sukurti tiesiog du $Z$ bozonai, kurie nėra
Higso bozono skilimo produktai, tačiau užfiksuotame vaizde gali atrodyti taip pat (užfiksuoti
keturi leptonai) \cite{HiggsEX}.
Tokiu atveju sakome, kad Higso bozono įvykis yra signalas, o dviejų $Z$ bozonų įvykis yra
triukšmas (angl. \textit{background}), į kurio egzistavimą reikia atsižvelgti.

Taip pat protonų susidūrimo metu dėl stipriosios sąveikos efektų gali susidaryti dalelių
srautai, vadinami čiurkšlėmis (angl. \textit{jets}).
Pasitaiko atvejų, kai detektoriuje užfiksuotos čiurkšlės yra neteisingai atpažįstamos kaip
kitos dalelės (pavyzdžiui, elektronas).
Taigi, įvykiai, kurių metu susidaro čiurkšlės taip pat gali būti triukšmo įvykiais net tuo
atveju, kai signalo proceso galutinėje būsenoje ieškomi tik leptonai.

Šiame darbe signalu yra laikomas Drell-Yan procesas: ieškoma tokių įvykių, kurių galutinė būsena
susideda iš dviejų užfiksuotų leptonų -- elektronų arba miuonų
	\footnote{Čia turima omenyje, kad bus užfiksuota elektrono-pozitrono pora,
	arba miuono-antimiuono pora.
	Trumpumo sumetimais didelių energijų fizikoje dalelės ir antidalelės dažnai vadinamos tuo
	pačiu (dalelės) vardu.
	Taip jos toliau bus vadinamos ir šiame darbe.}. 
Dviejų taonų galutinė būsena nebuvo tiriama dėl gan smarkiai besiskiriančios tyrimo metodikos
(taonai skyla į kitas daleles nepasiekę detektoriaus). Įdomu tai, kad šiuo atveju tiek tiriant
miuonų, tiek elektronų galutines būsenas, Drell-Yan proceso taonų kanalas
	\footnote{Skirtingos Drell-Yan proceso galutinės būsenos yra vadinamos kanalais:
	elektronų kanalas, miuonų kanalas ir taonų kanalas.}
yra triukšmas. Kiti pagrindiniai triukšmai yra šie: viršūninių kvarkų (angl. \textit{top quark})
poros įvykiai (sutrumpintai bus žymima $t\bar{t}\,$), dviejų bozonų įvykiai ($WW$, $WZ$ ir $ZZ$),
vieno viršūninio kvarko ir $W$ bozono įvykiai ($tW$), $W$ bozono ir čiurkšlės įvykiai ($\WJets$),
bei stipriosios sąveikos nulemti keletos čiurkšlių įvykiai (sutrumpintai vadinami $\QCD$).
Paskutiniai du iš išvarintų triukšmų čia yra todėl, kad čiurkšlė gali būti klaidingai atpažinta.

\subsubsection{Įvykių skaičiaus įvertinimas}

Kaip buvo rašoma ankstesniame skyrelyje, norint palyginti eksperimento metu užfiksuotą
statistiką su teoriškai numatomu rezultatu, svarbu yra įvertinti, kaip smarkiai užregistruoti
pasiskirstymai yra užteršti triukšmų.
Kadangi didelių energijų fizikos eksperimentinių duomenų analizė prasideda nuo tam tikrus
reikalavimus atitinkančių įvykių skaičiaus skaičiavimo, tai yra taikomi įvairūs metodai
tokius reikalavimus atitinkančių įvykių skaičiaus numatymui.
Tada galima lyginti, ar išmatuotas rezultatas atitinka numatytąjį.
Didelių energijų fizikoje pagrindinai yra naudojamos dvi metodikos: Monte Carlo (MC) modeliavimas
(tiek signalui, tiek triukšmui įvertinti) ir matavimu grįsti metodai (triukšmui įvertinti).

\subsubsection*{Monte Carlo modeliavimas}

Norint gauti teorinį eksperimento rezultatų įvertį galima kiek norima kartų atlikti virtualų
eksperimentą.
Kad tokį eksperimentą būtų galima atlikti, reikia turėti teoriją, kuri numato tam tikro mus
dominančio įvykio tikimybės pasiskirstymą, bei turėti atsitiktinių skaičių generatorių.
Generuojant atsitiktinius skaičius pagal tam tikrą tikimybės pasiskistymą galima sumodeliuoti
įvykius, su kuriais darant tas pačias analizės procedūras, kaip ir su eksperimento metu
užregistruotais įvykiais, galime gauti rezultatą, palyginamą su realybėje išmatuotu
rezultatu -- teorinį įvykių skaičiaus įvertį.

Aišku, realybėje didelių energijų protonų sklaidos eksperimentų modeliavimas nėra toks
tiessmukas.
Įprastiniu atveju, Didžiajame hadronų greitintuve vykstančių protonų susidūrimo įvykių
modeliavimas vykdomas keliais etapais.
Pirmiausia sumodeliuojamas pats protonų susidūrimas, kitaip tariant, sugeneruoti
atsitiktiniai skaičiai transformuojami į įvykio rezultatą -- kokios dalelės susidarė, bei
kokie jų parametrai (koordinatė, greitis ir pan.).
Po to modeliuojamas susidariusių dalelių sklidimas ir sąveika su medžiaga -- dalelių
detektoriaus komponentais.
Taip sumodeliuojama, ką galėjo užregistruoti detektorius įvykus būtent tokiam įvykiui.
Galiausiai gaunamas rezultatas, kuris savo išvaizda nesiskiria nuo tikro eksperimento metu
užregistruotų duomenų.
Tokiu būdų eksperimentatoriams yra labai patogu lyginti realiai gautą rezultatą su teoriniu.

CERN Kompaktiškojo miuonų solenoido (angl.\ \textit{Compact Muon Solenoid} -- CMS) eksperimente
įvykių modeliavimui dažniausiai naudojami šie programiniai paketai:
\begin{itemize}
	\item \textsc{Pythia8} -- tai C++ kalba parašytas programinis paketas, skirtas modeliuoti
	didelių energijų dalelių susidūrimus.
	\textsc{Pythia} paskirtis labai plati -- ji tinkama modeliuoti praktiškai bet kokiems žinomiems protonų
	susidūrimo metu vykstantiems procesams, joje yra įrašyta daug skiritngų teorinių modelių,
	partonų pasiskirstymo funkcijų.
	Pagrindinis \textsc{Pythia} trūkumas -- įvykiai modeliuojami tik nulinės eilės (angl.\
	\textit{leading order} -- LO) perturbacijų tikslumu \cite{pythia82}.
	
	\item FEWZ -- tai Fortran pagrindo programinis paketas, skirtas modeliuoti  $W$ ir $Z$
	bozonų susidarymą hadronų susidūrimo metu, taip pat šių bozonų skilimus į leptonus ir pan.
	FEWZ taip pat palaiko nemažai skirtingų partonų pasiskirstymo funkcijų bei modeliuoja įvykius
	antros eilės perturbacijų tikslumu \cite{fewz}.
	
	\item \textsc{Geant4} -- tai C++ pagrintdo programinis paketas, skirtas modeliuoti dalelių sąveiką
	su medžiaga.
	\textsc{Geant4} yra tinkama modeliuoti praktiškai bet kokių stabilių dalelių sąveikai su kuriuo nors iš
	realaus dalelių detektoriaus komponentų, taip pat dalelių trekų bei pataikymų į kitus detektoriaus
	komponentus užfiksavimui \cite{geant4}.
	Kompaktiškojo miuonų solenodo eksperimente šis programinis paketas naudojamas viso CMS detektoriaus
	atsako į konkretų (jau sumodeliuotą naudojantis kita programine įranga) protonų susidūrimo įvykį
	modeliavimui.
	
	\item CMSSW (\textit{CMS SoftWare}) -- tai CERN CMS eksperimente visuotinai naudojamas programinės
	įrangos rinkinys, skirtas ne tik įvykių modeliavimo, bet taip pat ir  visiems reikiamiems duomenų
	gavimo, apdorojimo ir analizės darbams atlikti.
	CMSSW išnaudoja bendrą CMS duomenų formatą -- įvykių duomenų modelį (angl.\ \textit{Event Data Model}
	-- EDM, kuris tinkamas skirtingiems duomenų apdorojimo lygmenims bei analizės etapams.
	Joje yra tiek protonų susidūrimui, tiek detektoriaus atsakui modeliuoti skirti programiniai paketai
	(tokie, kaip jau minėti \textsc{Pythia8} ir \textsc{Geant4})
\end{itemize}

Vis dėlto, neretai modeliuotas įvykių skaičiaus įvertis nebūna pakankamai tikslus dėl
neapibrėžtumų, kuriuos įneša atskirų triukšmo procesų įvykių tikimybės, neidealus detektoriaus
atsako modeliavimas ir pan.
Šių problemų gali būti išvengiama triukšmo įvykių skaičiaus įvertinimui naudojant matavimu
grįstus metodus.

\subsubsection*{Matavimu grįsti metodai}

Matavimu grįsti triuškmo įvykių skaičiaus įvertinimo metodai apima ir matavimą, ir modeliavimą.
Su skirtingos specifikos procesais siejamų įvykių skaičiui įvertinti naudojami skirtingi
matavimu grįsti metodai, tačiau jie visi remiasi labai panašia ideologija.
Pagrindinis matavimu grįstų įvykių skaičiaus įvertinimo metodų principas remiasi signalo ir
kontrolinės sričių apibrėžimais.
Signalo sritis (angl. \textit{signal region}) -- tai fazinės erdvės dalis, apribota tokiais
parametrais, kad į ją patektų kuo didesnė dalis signalo įvykių (tačiau neišvengiamai patenka
ir kažkiek triukšmo įvykių).
Kontrolinė sritis (angl. \textit{control region}) -- tai fazinės erdvės dalis, apribota
tokiais parametrais, kad į ją patektų kuo daugiau triukšmo įvykių ir, idealiu atveju, signalo
įvykių nepatektų išvis.
Kiekvienas matavimu grįstas metodas turi savo specifinę operaciją, kuria triukšmo įvykių skaičius,
išmatuotas kontrolinėje srityje, būna transformuojamas į triuškmo įvykių skaičių, patenkantį į
signalo sritį:

\begin{equation}
	N_{\mathrm{Bkg}}^{\mathrm{Signal}} = f( N_{\mathrm{Bkg}}^{\mathrm{Control}} )
	\label{eq:data-driven}
\end{equation}

Vieni iš populiariausių matavimu grįstų metodu yra klaidingo atpažinimo (angl.
\textit{fake rate} -- FR) ir ABCD metodai.
Jie naudojami įvertinti skaičiui tokių triukšmo įvykių, kurių metu susidarė čiurkšlės.
Tokių triukšmo įvykių, kurių metu gali susidaryti keli vienodų arba skirtingų
rūšių leptonai, skaičiui įvertinti naudojamas $\emu$ metodas, kuris ir buvo naudojamas šiame
darbe.

\paragraph{$e\mu$ metodas\\}

$\emu$ metodas -- tai matavimu grįstas triukšmo įvykių skaičiaus įvertinimo metodas, kuris
naudojamas dviejų vienodų leptonų (elektronų arba miuonų) galutinės būsenos triukšmams
įvertinti, pasinaudojant įvykių, kuriuose buvo užfiksuoti skirtingi leptonai (elektronas
ir miuonas), duomenimis.
Šis metodas tinkamas įvertinti tokiems triukšmams, kurie siejami su dviejų nestabilių
dalelių, kurios turi galimybę nepriklausomai viena nuo kitos skilti į leptonus, susidarymu.
Tokie triukšmo procesai yra: $\WW$, $\WZ$, $\ZZ$, $tW$, $\tbarW$, $\ttbar$, $\DYtau$.
Galima nesunkiai matyti, kad visuose šiuose procesuose figūruoja dvi sunkios dalelės, kurios
gali skilti leptoniškai.

Kaip pavyzdį $\emu$ metodo veikimui paaiškinti galime paimti $WW$ procesą.
Jeigu nagrinėjame tik $W$ bozono skilimus į elektroną arba miuoną (kartu su atitinkamais
neutrinais), tai turėdami įvykį, kurio metu susidaro du $W$ bozonai, galime turėti
keturias skirtingas galutines būsenas: $WW \! \rightarrow \! e^+e^-$,
$WW \! \rightarrow \! e^+\mu^-$, $WW \! \rightarrow \! \mu^+e^-$ ir
$WW \! \rightarrow \! \mu^+\mu^-$.
Čia dešinėje rodyklės pusėje nerašėme atitinkamų neutrinų, nes $\emu$ metodo skaičiavimuose
jie nedalyvauja.
Pasinaudodami Monte Carlo modeliavimu galime įsivertinti $ee$ ir $\emu$ įvykių skaičiaus
santykį $N_{ee}^{\mathrm{MC}} / N_{\emu}^{\mathrm{MC}}$, nekreipdami dėmesio į elektrono
ir miuono elektrinius krūvius, tik reikalaudami, kad jie būtų skirtingi.
Idealiu atveju šis santykis turėtų būti apytiksliai lygus $1:2$, nes tiek elektrono, tiek
miuono masės santykis su motininės dalelės mase yra labai mažas.
Tai reiškia, kad nestabili dalelė gali skilti į miuoną arba elektroną su apytiksliai
vienoda tikimybe.
Padarius prielaidą, kad šis santykis turėtų būti vienodas tiek eksperimentiniuose, tiek
modeliuotuose duomenyse, galime įvertinti (angl.\ \textit{estimate}), koks turėtų būti su
anksčiau minėtais procesais susijusių triukšmo įvykių skaičius $ee$ duomenyse:

\begin{equation}
	N_{t\bar{t}}^{ee , \; \mathrm{Est}} =
	\frac{ N_{t\bar{t}}^{ee , \; \mathrm{MC}} }{ N_{t\bar{t}}^{e\mu , \; \mathrm{MC}} }
	\cdot N_{t\bar{t}}^{e\mu , \; \mathrm{Data}} \; .
	\label{eq:emuMethod}
\end{equation}

Vis dėlto, ši išraiška yra idealizuota, nes realybėje neįmanoma iš eksperimentinių duomenų
vienareikšmiškai išskirti, kurie įvykiai yra sietini su konkrečiu procesu (šiuo atveju --
$\ttbar$), todėl formulę reikia taikyti visiems $\emu$ procesams iškart:

\begin{equation}
	N_{ee}^{\mathrm{Įv.}} =
	\frac{ N_{ee}^{\mathrm{MC}} }{ N_{\emu}^{\mathrm{MC}} }
	\cdot N_{\emu}^{\mathrm{Data}} \; .
	\label{eq:emuReal}
\end{equation}

Prisiminus \eqref{eq:data-driven} iraišką ir palyginę ją su \eqref{eq:emuReal} galime
sakyti, kad $N_{t\bar{t}}^{ee , \; \mathrm{Data}}$ yra signalo sritis, o
$N_{t\bar{t}}^{e\mu , \; \mathrm{Data}}$ yra kontrolinė sritis. Joje yra vien tik triukšmo
įvykiai.
Triukšmo įvykių skaičiaus signalo srityje įvertį gauname su triukšmo įvykių skaičiumi
kontrolinėje srityje atlikę paprastą matematinę operaciją -- padauginę jį iš modeliuotų
$ee$ ir $\emu$ įvykių skaičiaus santykio.


\subsubsection{Modeliuotų įvykių skaičiaus normavimas}\label{sec:MCweight}

Jeigu kokio nors mus dominančio proceso tikėtinumą aprašo jo skerspjūvis $\sigma$, tai atlikus
$\Lumi$ integruotą šviesį atitinkančių protonų susidūrimų mus dominanti įvykis teoriškai turėtų
įvykti $N_{0}$ kartų. Pagal \eqref{eq:evno} formulę:
\begin{equation}
	\sigma \Lumi=N_{0} \; .
	\label{eq:ilumi}
\end{equation}

Išmatuotos mus dominančių įvykių statistikos payginimui su teorija galime susimodeliuoti tiek mus
dominantį procesą atitinkančių įvykių, kiek jų norime, arba kiek leidžia turimi resursai.
Bendru atveju sumodeliuotų įvykių skaičius $N$ nesutampa su eksperimente užregistruotų įvykių
skaičiumi $N_0$, todėl modeliuotus įvykius reikia sunormuoti, kad atitiktų eksperimentinį
integruotą šviesį.
Tai yra daroma kiekvienam modeliuotam įvykiui priskiriant po tam tikrą svorį $\omega$:

\begin{equation}
	\sum_{i=1}^{N}1\cdot \omega_{i}=N_{0} \; .
	\label{eq:wsum}
\end{equation}

Paprastčiausiu atveju visiems modeliuotiems įvykiams, kurie yra siejami su tuo pačiu procesu,
galima priskirti vienodus svorius. Tokiu atveju \eqref{eq:wsum} galima perrašyti:

\begin{equation}
	\sum_{i=1}^{N} 1 \cdot \omega_{i} = N \omega \; ,
	\label{eq:swsum}
\end{equation}

o kiekvieno įvykio svoris apskaičiuojamas taip:

\begin{equation}
	\omega = \frac{ N_{0} }{ N } = \frac{ \sigma \Lumi }{N} \; .
	\label{eq:weight}
\end{equation}

Kai naudojama programinė įranga geba modeliuoti įvykius aukštesniu, nei nulinės eilės
perturbacijų tikslumu, neretai skirtingiems įvykiams pats įvykių generatorius jau iškart
gali priskirti skirtingus svorius.
Tokiu atveju \eqref{eq:weight} formulė nebetinka, o ir bendras modeliuotų įvykių skačiaus
apskaičiavimas nebėra toks trivialus.
Tokiu atveju bendras įvykių skaičius $N$ modeliuotame duomenų rinkinyje yra apskaičiuojamas
kaip visų įvykių svorių suma, kuri nėra lygi fiziniam ten esančių įvykių skaičiui $N_{\mathrm{ev}}$:

\begin{equation}
	N = \sum_{i=1}^{N_{\mathrm{ev}}} \omega_{i} \; ,
\end{equation}

o svorinis daugilis, priskiriamas kiekvienam įvykiui apskaičiuojamas taip:

\begin{equation}
	\omega_{i}^{\mathrm{Piln.}} = \omega_{i} \frac{ \sigma\Lumi }{ \sum_{i=j}^{N_{\mathrm{ev}}}\omega_{j} } =
	\omega_{i} \frac{ \sigma\Lumi }{N} \; .
	\label{eq:NLOweight}
\end{equation}

\subsubsection{Didysis hadronų greitintuvas ir Kompaktiškasis miuonų solenoidas}

Šveicarijos - Prancūzijos pasienyje esantis Europos branduolinių tyrimų organizacijai CERN
priklausantis Didysis hadronų priešpriešinių srautų greitintuvas yra didžiausias ir galingiausias
dalelių greitintuvas pasaulyje.
Maždaug $27$ km perimetro žiedinis geritintuvas slepiasi apytiksliai $100$ m gylyje po žeme.
Nors didžiajame hadronų greitintuve galima vykdyti įvairių hadroninių dalelių (pavyzdžiui, švino
branduolių) susidūrimus, dažniausiai ten vykdomi ir daugiausiai tiriami yra protonų susidūrimai.
Dalelės, prieš patekdamos į šį greitintuvą praeina kelias greitinimo pakopas kituose mažesniuose
greitintuvuose, kurie anksčiau buvo naudojami vykdyti dalelių susidūrimams \cite{accelerators}.
Nuo $2015$ metų Didžiajame hadronų greitintuve vykstančių protonų susidūrimų energija sieka net $13$ TeV.
Įprastai jie vyksta kas $25$ ns keliuose skirtinguose žiedo taškuose, aplink kuriuose yra išdėstyti dalelių
detektoriai, priklausantys skirtingų eksperimentų grupėms.
Dvi didžiausios ir geriausiai žinomos eksperimentinės grupės yra CMS ir ATLAS \cite{LHCexperiments}.

Kompaktiškasis miuonų solenoidas (angl.\ \textit{Compact Muon Solenoid} -- CMS) yra plačios paskirties
detektorius, sukurtas įvairių skirtingų dalelių detektavimui.
Dėl šios priežasties CMS eksperimento mokslininkai gali vykdyti labai skirtingų tematikų tyrimus,
pavyzdžiui, tikslinti ir tikrinti Standartinį modelį, ieškoti naujų dalelių arba net papildomų
dimensijų ar tamsiosios medžiagos \cite{aboutCMS}.

CMS yra cilindrinės geometrijos detektorius, jo aukštis ir plotis -- apytiksliai po $15$ m, o ilgis --
apie $21$ m. Detektorius vadinamas kompaktiškuoju todėl, kad kaip tokių matmenų jis yra labai masyvus.
Jo masė -- apytiksliai $14000$ tonų.
Detektorius susideda iš daug sluoksnių ir segmentų, kurie skirti detektuoti skirtingų rūšių dalelėms.
CMS \ltq{širdis} -- didžiausias pasaulyje solenoidinis elektromagnetas.
Tai iki superlaidumo temperatūros atšaldoma ritė, kuria darbo metu teka maždaug $18.5$ A stiprio
elektros srovė ir kurios viduje sukuriamas $4$ T magnetinis laukas \cite{CMSdetector}.

%senas
CMS detektoriaus segmentus galima pamatyti \ref{fig:CMSslice} paveiksle.
Kiekvienas detektoriaus segmentas turi cilindrinę ir antgalių dalis, kurios yra sluoksniškai išdėstytos
einant tolyn nuo protonų susidūrimo vietos.
Kiekvienas subdetektorių sluosknis yra skirtingas ir turi savo paskirtį.

Arčiausiai aplink protonų spindulį yra išdėstytas trekų detektorius (angl.\ \textit{silicon tracker}),
pagamintas iš silicio pikselių ir juostelių.
Į trekų detektorių pataikiusios elektringos dalelės išlaisvina krūvininkus taip sugeneruodamos elektrinį
signalą.
Taip iš užfiksuoto signalo keliose skirtingose trekų detektoriaus dalyse galima nustatyti, kokia kryptimi
nulėkė protonų susidūrimo metu sukurtos elektringosios dalelės.

Tolimesnis sluoksnis einant nuo protonų susidūrimo taško yra elektromagnetinis kalorimetras (angl.\
\textit{Electromagnetic Calorimeter} -- ECAL).
Šio subdetektoriaus paskirtis -- detektuoti elektronus ir fotonus.
Svarbiausia elektromagnetinio kalorimetro sudedamoji dalis -- scintiliatorius švino volframatas $\mathrm{PbWO}_{4}$,
kuris ima švytėti, kai į jį pataiko energingas elektronas arba fotonas.
Šios dalelės dažniausiai čia praranda visą savo kinetinę energiją, kuriai sukelto švytėjimo intensyvumas
būna proporcingas.
Dėl šios priežasties elektromagnetiniu kalorimetru galima išmatuoti, su kokia energija dalelė išlėkė iš
protonų susidūrimo vietos \cite{Ecal}.
Elektroną nuo fotono galima atskirti pagal tai, ar galima susieti kalorimetre užfiksuotą signalą su trekų
detektoriuje užregistruotu treku.

Dar toliau nuo protonų spindulio yra išsidėstęs hadronų kalorimetras (angl.\ \textit{hadron calorimeter} -- HCAL).
Šio subdetektoriaus veikimo principas panašus į elektromagnetinio kalorimetro, tik šis skirtas detektuoti
hadronus bei išmatuoti jų energiją.
Tačiau yra ir didesnių skirtumų: čia naudojamas plastiko scintiliatorius, bei, kadangi hadronai yra gerokai
sunkesni ir skvarbesni, norint juos sustabdyti reikia išnaudoti stipriąją sąveiką.
Dėl šios priežasties hadronų kalorimetre tarp scintiliatoriaus sluoksnių yra įterptos vario plokštės, taip
pat šis subdetektorius yra gerokai storesnis už elektromagnetinį kalorimetrą \cite{Hcal}.

Už hadronų kalorimetro yra jau minėtas superlaidus solenoidas.
Jo paskirtis -- iškreivinti krūvį turinčių dalelių trajektrorijas.
Pagal tai, kokia kryptimi užsisuka dalelės lėkio kelias, galima nustatyti, ar dalelės elektrinis krūvis buvo
teigiamas, ar neigiamas, bei nustačius, kokio tipo dalelė buvo užfiksuota, iš jos trajektorijos kreivumo
spindulio galima nustatyti jos impulsą.
Taip pat solenoidas veikia kaip papildomas stabdis skvarbiausiems hadronams.
Vis dėlto, norint užtikrinti, kad kuo mažiau hadronų pasiektų dar tolimesnius detektoriaus sluoksnius, už
solenoido yra sumontuotas dar vienas papildomas hadronų kalorimetro sluoksnis, vadinamas išoriniu hadronų
kalorimetru (angl.\ \textit{HCAL outer}).

Išorinį hadronų kalorimetrą dar supa labai daug dėmesio sulaukianti detektoriaus dalis (tai ryškiai atsispindi
ir detektoriaus pavadinime) -- keliais sluoksniais išdėstyti miuonų detektoriai.
Nors miuonų detektorių sistemoje yra išnaudojami kelių skirtingų tipų detektoriai, jie visi yra priskiriami
tai pačiai dujinių detektorių kategorijai.
Miuonų detektoriai yra išrikiuoti toliausiai nuo protonų susidūrimo taško, nes jie yra apie $200$ kartų
sunkesni už elektronus, bei nesąveikauja stipriąja sąveika.
Dėl šios priežasties jie yra nesustabdomi nei elektromagnetiniame, nei hadronų kalorimetre.
Iš tiesų, jie nėra sustabdomi nei miuonų detektorių sistemoje -- dujiniai detektoriai tik užfiksuoja jų
trajektoriją.
Miuonų impulsas yra nustatomas iš trajektorijos kreivumo \cite{MuonChambers}.
Kad tai būtų galima nustatyti kuo efektyviau, tarp miuonų detektorių yra sumontuotos geležinės magnetinio
lauko apgrąžos plokštės, kurios sustiprina už solenoido esantį magnetinį lauką ties miuonų detektoriais,
tuo pačiu neleisdamos jam tęstis toli už detektoriaus.
Taip pat šios plokštės užblokuoja kelią paskutinėms iki jų prasiskverbusioms dalelėms, kurios nėra miuonai
arba neutrinai \cite{Solenoid}.

\begin{figure}[H] \centering
	\includegraphics[width=0.95\textwidth]{CMSslice_LT.png}
	\caption{\label{fig:CMSslice}Skersinis CMS detektoriaus pjūvis \cite{CMSslice}.
	Išorinis hadronų kalorimetras nepavaizduotas. Skirtingos linijos žymi skirtingų dalelių, išlekiančių
	iš protonų susidūrimo vietos, trajektorijas.
	Trūki linija žymi elektriškai neutralios dalelės trajektoriją (kuri silicio trekų detektoriuje
	neužfiksuojama.}
\end{figure}

\subsubsection{Protonų susidūrimų atkūrimas}\label{sec:ppReco}

Po protonų susidūrimo susidariusios dalelės detektoriuose aptinkamos netiesiogiai, t.y., realiai  yra
užregistruojamas tik tam tikro pobūdžio signalas detektoriaus komponentuose, pagal kurį galima spręsti,
kad į jį pataikė tam tikra dalelė su tam tikra energija ir pan.
Tam, kad būtų galima gauti tam tikrą dalelę aprašančius fizikinius dydžius reikia gerai išmanyti, ką
ir kaip detektorius detektuoja.
O norint susidaryti pilną protonų susidūrimo vaizdą dar reikia apjungti visų detektoriaus komponentų
užregistruotą informaciją.
Šiam tikslui įgyvendinti yra naudojama sudėtingi programiniai algoritmai, iš kurių keletas bus trumpai
aprašyti.


\subsubsection*{Dalelių srautas}

Dalelių srautas (angl.\ \textit{particle flow} -- PF) -- tai įvykių atkūrimo algoritmas, savo tikslui pasiekti
naudojantis visų CMS subdetektorių užregistruotą informaciją.
Šis algoritmas naudoja iteracinę įvykių atkūrimo metodiką: pirmiausia naudodamasis smarkiai suvaržytais
atrankos kriterijais išrenka tas trajektorijas, dėl kurių yra \ltq{užtikrinčiausias}, bei apskaičiuoja
su jomis susijusius svarbius dydžius (pavyzdžiui, dalelės krūvį, skersinį impulsą, pseudospartą,
kokia dalelė apskritai susidarė ir pan.).
Po to šios atrinktos trajektorijos yra išimamos iš pilno duomenų bloko ir sekančioje iteracijoje iš
naujo bandoma išrinkti kitas trajektorijas šiek tiek sušvelninus atrankos kriterijus.
Taip trajektorijų atrinkimo procesas kartojamas kol panaudojama visa subdetektorių informacija.
Dalelių srauto įvykio atkūrimo rezultatas -- duomenys, kurie iš pažiūros atrodo taip pat kaip po
Monte Carlo protonų susidūrimo modeliavimo.
Vis dėlto, kadangi šis algoritmas negali visada dalelių atpažinti idealiai tiksliai, dažniausiai
atkurtos dalelės vadinamos dalelėmis kandidatėmis.

Dar dalelių srauto algoritmas apskaičiuoja nemažai daliai CMS vykdomų tyrimų svarbų dydį -- dalelės
kandidatės treko atskirumą (angl.\ \textit{isolation}).
Šis dydis apibrėžiamas kaip dalelių, patekusių į tam tikro pločio kūgį, nubrėžtą aplink tiriamosios
dalelės trajektoriją, skersinių impulsų suma.
Įprastai norima, kad ta suma būtų kuo mažesnė, arba, kitaip tariant, dalelės srautas kuo atskiresnis.
Tada norint būti labiau užtikrintam, kad, pavyzdžiui miuono kandidatas tikrai yra miuonas, galima
uždėti apribojimą, reikalaujanti, kad miuono kandidato askirumas neviršytų tam tikro dydžio
(pavyzdžiui, $15\%$ paties miuono skersinio impulso vertės).

Norint patikslinti atskirumo skaičiavimo vertę neretai būna panaudojama pataisa, kuri atsižvelgia į
pašalinius protonų susidūrimus.
Kadangi protonų spinduliai detektoriuje susiduria kas $25$ ns (kol visos susidariusios dalelės
pasiekia detektorius gali įvykti dar keli nauji susidūrimai), bei vieno spindulių susidūrimo metu
dažniausiai susiduria daugiau negu viena protonų pora, tai svarbu yra atskirti, ar stebimos dalelės
yra susidariusios būtent mus dominančio protonų susidūrimo metu.
Dėl šios priežasties į atskirumo skaičiavimą būna įskaitomi dalelių, susidariusių pašalinių
protonų susidūrimo metu, skersiniai impulsai $p_{\mathrm{T}}^{\mathrm{PU}}$ (PU -- nuo angliško
termino \textit{pile-up}).
Santykinis leptonų kandidatų atskirumas apskaičiuojamas pagal tokią formulę:

\begin{equation}
	\label{eq:isolation}
	I^{\mathrm{rel.}}_{\mathrm{PF}} = \frac{1}{p_{\mathrm{T}}} 
	\left[ \sum_{\Delta R<0.3} p_{\mathrm{T}}^{\mathrm{hadron^{\pm}}} +
	\mathrm{max} \left( \sum_{\Delta R<0.3} p_{\mathrm{T}}^{\mathrm{hadron^0}} + 
	\sum_{\Delta R<0.3} p_{\mathrm{T}}^{\gamma} -
	\Delta \beta \sum_{\Delta R<0.3} p_{\mathrm{T}}^{\mathrm{PU}}
	\, ,\;\; 0 \right) \right] \; \mathrm{,}
\end{equation}

čia $\Delta R = \sqrt{\Delta \phi^{2} + \Delta \eta^{2}}$ -- kūgio, brėžiamo aplink leptono kandidato
trajektoriją, plotis, $p_{\mathrm{T}}^{\mathrm{hadron^{\pm}}}$ -- elektringų hadronų skersiniai impulsai,
$p_{\mathrm{T}}^{\mathrm{hadron^0}}$ -- elektriškai neutralių hadronų skersiniai impulsai,
$\Delta\beta=0.5$ -- iš modeliavimo ivertintas koeficientas, apytiksliai lygus protonų susidūrimo metu
susidarančių elektriškai neutralių ir krūvį turinčių hadronų skaičiaus santykiui.


\subsubsection*{CMSSW}

Kadangi CMS eksperimente naudojamas bendras plačios paskirties EDM duomenų formatas, tai įvykių atkūrimas
būna atliekamas naudojant tam pritaikytą CMSSW įvykių atkūrimo biblioteką.
Ja naudojantis galima vykdyti skirtingų lygių įvykių atkūrimą, pasirinkti norimus įvykių atkūrimo
parametrus ir pan.
CMSSW taip pat išnaudoja ir dalelių srauto algoritmą.
Po įvykių atkūrimo iš RAW formato duomenų failo (skirtingų detektoriaus komponentų užregistruota informacija)
gaunamas RECO formato duomenų failas (išsami atkurto įvykio informacija), kuriuos iš principo
jau gali naudoti įvykių analizei.
Vis dėlto, kadangi RECO formato duomenyse saugoma labai daug informacijos, kuri ne visa reikalinga tolimesnei
analizei, duomenis analizuojantys mokslininkai dažniau naudoja sumažintus duomenų formatus AOD ir MiniAOD.


\subsubsection{Trigeriai}\label{sec:trigger}

Norint užregistruoti kuo daugiau mažo tikėtinumo įvykių protonų susidūrimai (dažniausiai) yra vykdomi kas
$25$ ns.
Su šiuolaikinėmis technologijomis išsaugoti kiekvieno susidūrimo informaciją yra praktiškai neįmanoma.
Taip pat, didžioji dalis protonų susidūrimų nebūna pakankamai energingi, arba jų metu įvyksta didžiausio
tikėtinumo įvykiai, kuriuos laikome jau pakankamai ištirtais.
Dėl šių priežasčių yra naudojamos kompiuterinės sistemos, vadinamos trigeriais.
Trigerių paskirtis -- iš prieinamos su įvykiu susijusios informacijos nuspręsti, ar įvykis vertas dėmesio.
Išrašinėjant tik tuos įvykius, kurie aktyvavo trigerį, pasiekiamas toks įvykių išsaugojimo dažnis, su kuriuo
naudojama elektronika jau gali susitvarkyti, o taip pat ir sutaupoma vietos.
Trigeriai būna kelių skirtingų lygių, turimos sistemos būtų išnaudojamos kaip įmanoma efektyviau.

\textbf{Pirmojo lygio trigeris} -- tai arčiausiai detektoriaus sumontuota superkompiuterių sistema.
Joje realiu laiku minimaliai apdorojami kalorimetrų bei miuonų sistemos duomenys ir esant tam tikram iš
anksto užprogramuotam rezultatui trigeris aktyvuojasi.
Tada duomenys būna išsaugomi ir siunčiami toliau kitam analizės lygiui.
Pirmojo lygio trigeris iš $4 \! \cdot \! 10^{7}$  įvykių per sekundę palieka tik apie $10^{5}$ \cite{HLtrigger}.

\textbf{Aukšto lygio trigeris} -- tai superkompiuterių sistema, į kurią atkeliauja pirmojo lygio trigerį
aktyvavę įvykiai.
Čia įvykiai atkuriami jau pasinaudojant ir trekų detektoriaus užregistruotais duomenimis, bei naudojami
griežtesni atrankos kriterijai.
Tai padeda įvykių skaičių sumažinti iki maždaug $800$ įvykių per sekundę ir įrašomi ilgalaikiam saugojimui,
kad vėliau galėtų būti tyrinėjami mokslininkų.

Šiame darbe buvo naudojami tokie aukšto lygio trigeriai:
\begin{enumerate}
\item Dviejų elektronų trigeris, kuris aktyvuojamas tada, kai aptinkami bent du elektronai, vieno iš kurių skersinis
impulsas didesnis, nei $23$ GeV, o kito -- didesnis nei $12$ GeV.
\item Vieno miuono trigeris, kuris aktyvuojamas tada, kai aptinkamas bent vienas miuonas su skersiniu impulsu,
didesniu, nei $24$ GeV. Miuonas gali būti atpažintas pasinaudojant tiek trekų detektoriaus, tiek miuonų
detektoriaus informacija.
\end{enumerate}


\subsubsection{Įvykių atranka}\label{sec:selection}

Kadangi trigeriai turi veikti greitai, jie dažniausiai būna nesudėtingi, tai yra, jų aktyvavimą nulemia
nedidelis parametrų skaičius, taip pat jie naudojami įvykiui dar nesant pilnai atkurtam.
Norint daryti aukštesnio lygio duomenų analizę neužtenka tik atsirinkti įvykius, kurie aktyvavo tam tikrą
aukšto lygio trigerį.
Reikia panaudoti daugiau papildomų atrankos kriterijų, kad liktų tik tokie užregistruoti protonų susidūrimo
įvykiai, kurie yra panašiausi į su tiriamu procesu sietinus įvykius.
Akivaizdu, kad tiriant skirtingas tam tikro proceso galutines būsenas reikia taikyti ir skirtingus atrankos
kriterijus.
Taigi, $ee$ ir $\mumu$ įvykiai, o taip pat ir $e\mu$ įvykiai, naudoti įvertinant Drell-Yan proceso $ee$ ir
$\mumu$ galutinių būsenų triukšmo įvykių skaičių, buvo atrenkami šiek tiek skirtingai.

Dviejų elektronų galutinės būsenos įvykiams buvo taikomi šie atrankos kriterijai:
\begin{enumerate}
	\item Aktyvuotas ankstesniame skyrelyje minėtas dviejų elektronų trigeris;
	\item Užregistruoti lygiai du elektronai, atitinkantys \textit{MediumID} reikalavimus.
	\item Abu užregistruoti elektronai turi būti pataikę į elektromagnetinio kalorimetro segmentą, kurio
	pseudospartos absoliučioji vertė neviršija $2.4$.
	\item Abu užregistruoti elektronai neturėtų būti pataikę į elektromagnetinio kalorimetro cilindrinės ir
	antgalio dalies sankirtą, kurioje eina detektoriaus elektronikos laidai, sumažinantys detektavimo efektyvumą
	toje srityje. Ši vieta yra $1.4442<|\eta_{SC}|<1.566$ ruože.
	\item Greitesniojo iš dviejų užregistruotų elektronų skersinis impulsas turėtų būti didesnis už $28$ GeV, o
	lėtesniojo -- didesnis už $17$ GeV
\end{enumerate}

Dviejų miuonų galutinės būsenos įvykiams buvo taikomi tokie atrankos reikalavimai:
\begin{enumerate}
	\item Aktyvuotas ankstesniame skyrelyje minėtas vieno miuono trigeris;
	\item Užregistruoti bent du miuonai, atitinkantys \textit{TightID} reikalavimus. Vienas iš šių miuonų turi būti
	tas, kuris aktyvavo	minėtą trigerį.
	\item Miuonų atskirumas, apskaičiuotas naudojantis dalelių srauto algoritmu turi neviršyti $15\%$ miuonų skersinio
	impulso vertės.
	\item Miuonų pseudospartos absoliutinė vertė turi neviršyti $2.4$.
	\item Iš visų minėtus kriterijus atitinkančių miuonų atrenkami du, kurių trajektorijas su mažiausiu neapibrėžtumu
	galima suvesti į vieną viršūnę.
	\item Iš likusių dviejų miuonų greitesniojo impulsas turi viršyti $28$ GeV, o lėtesniojo -- $17$ GeV.
	\item Kampas tarp dviejų miuonų lėkimo iš viršūnės krypčių turi būti ne didesnis, kaip $\pi - 0.005$ rad.
	\item Miuonai turi būti priešingų elektrinių krūvių.
\end{enumerate}

Elektrono ir miuono galutinės būsenos įvykiams buvo taikomas anksčiau aprašytų atrankos kriterijų junginys.
Tai yra, elektronui buvo taikomi tokie atrankos kriterijai, kokie buvo naudojami $ee$ įvykių atrankoje, o miuonui --
tokie, kokie buvo naudojami $\mumu$ įvykių atrankoje. Reikalavimai, kurie $\mumu$ atrankoje buvo taikomi miuonų poroms,
šiuo atveju buvo taikomi elektrono-miuono poroms.
Kadangi dviejų elektronų trigeris, naudotas $ee$ atrankoje, šiuo atveju buvo netinkamas, tai $\emu$ atrankoje buvo naudojamas
vieno miuono trigeris, aprašytas ankstensiame skyrelyje.

\subsubsection{Duomenų analizės biblioteka ROOT}

Atliekant šį darbą buvo naudojamasi CERN mokslininkų plačiai naudojamu programinės įrangos rinkiniu ROOT \cite{ROOT}.
ROOT yra didelių duomenų kiekių analizei skirta sistema, parašyta daugiausia C++ kalba.
Naudojantis ROOT galima atlikti įvairaus pobūdžio duomenų apdorojimą, analizę, grafinį atvaizdavimą, didelius duomenų
kiekius įrašinėti ir skaityti iš stuktūrizuotų failų ir pan.
Taip pat ROOT suteikia galimybę naudotis daugybe fizikiniams skaičiavimams naudingų klasių.
Atliekant primityvią duomenų analizę, galima ją daryti tiesiog vedant komandas ROOT aplinkoje bei naudojantis jos
suteikiama grafine sąsaja, tačiau dibant su didesniais duomenų kiekiais ir norint atlikti labiau struktūrizuotą analizę
reikia rašyti C++ programinius kodus su ROOT klasėmis ir juos leisti ROOT aplinkoje.
Būtent taip ir buvo atliekama duomenų analizė šiame darbe.

\subsubsection{Protonų susidūrimų duomenys}\label{sec:data}

Šioje analizėje naudoti 2016 metais CERN CMS detektoriuje užregistruoti $13$ TeV energijos protonų susidūrimų duomenys,
atitinkantys $35.9$ \invfb integruotąjį šviesį, bei įvairių procesų modeliuoti įvykiai buvo saugomi Pietų Korėjos KISTI
Tier3 duomenų centre.
Duomenys buvo laikomi \ttt{.root} formato failuose.
Bendra visų duomenų rinkinių apimtis -- apie $14$ TB.
Failuose informacija buvo saugoma ROOT medžiuose (\ttt{TTree} klasė) -- medžio \ltq{šakos} -- tai tam tikri protonų
susidūrimo įvykį aprašantys kintamieji (pavyzdžiui, užregistruotų elektronų, miuonų skaičius, jų skersiniai impulsai ir pan.),
o ant šakų esantys \ltq{lapai} -- tų kintamųjų vertės, užregistruotos kiekvienam įvykiui.
Duomenys skaitomi pasinaudojant ROOT \ttt{TChain} klase, kuri medžius sujungia į grandinę ir leidžia pasiekti kiekvieno įvykio
informaciją paeiliui.

\subsubsection{Statistinių paklaidų įvertinimas}\label{sec:uncertainties}

CMS detektoriuje užfiksuotų įvykių skaičius yra diskretinis dydis, aprašomas Puasono pasiskirstymu.
Yra parodoma, kad Puasono pasiskirstymu aprašomų įvykių skaičiaus standartinis nuokrypis savo skaitine verte yra lygus
kvadratiniai šakniai iš labiausiai tikėtino įvykių skaičiaus \cite{Poisson}.
Vis dėlto, šis dydis nėra žinomas, todėl eksperimento metu užfikstuotų įvykių skaičiaus neapibrėžtumo verte yra laikoma
kvadratinė šaknis iš to paties užregistruotų įvykių skaičiaus.

Kai įvykių skaičius nėra tiesiogiai išmatuotas dydis, o yra apskaičiuotas iš kelių kitų išmatuotų dydžių, turinčių savo
neapibrėžtumus (pavyzdžiui, įvykių skaičius, apskaičiuotas pagal \eqref{eq:emuMethod} formulę), tai apskaičiuotojo įvykių
skaičiaus neapibrėžtumas apskaičiuojamas pasinaudojant šia išraiška:

\begin{equation}
	\Delta f(x, y, ...) =
	\sqrt{ \left( \frac{\partial f}{\partial x} \Delta x \right)^{2} +
	\left( \frac{\partial f}{\partial y} \Delta y \right)^{2} + ... } \;\; \mathrm{.}
	\label{eq:DerUnc}
\end{equation}

Kaip buvo rašyta \ref{sec:MCweight} skyrelyje, modeliuoti įvykiai turi jiems priskirtus svorius, todėl sunormuotas modeliuotų įvykių
skaičius bendru atveju neatitinka fizinio sumodeliuotų įvykių skaičiaus.
Dėl šios priežasties sunormuoto modeliuotų įvykių skaičiaus neapibrėžtumas skaičiuojamas ne kaip kvadratinė šaknis iš
sunormuoto įvykių skaičiaus, o kaip kvadratinė šaknis iš visų modeliuotų įvykių svorių kvadratų sumos (svoriai sudedami
pagal Pitagoro teoremą):

\begin{equation}
	\Delta N^{\mathrm{MC}}_{\mathrm{norm.}} = \sqrt{\sum_{i=1}^{N^{\mathrm{MC}}}w_{i}^{2}} \; .
	\label{eq:Sumw2Unc}
\end{equation}

Galima pastebėti, kad, jeigu visi modeliuoti įvykiai turėtų vienetinius svorius, ši suma taip pat taptų lygi kvadratinei
šakniai iš įvykių skaičiaus, kaip ir skaičiuojant neapibrėžtumą eksperimentiniams įvykiams.

\clearpage
\section{Drell-Yan proceso analizė naudojant 2016 metų CERN CMS duomenis}

\subsection{Matavimo ir modeliavimo palyginimas}

\subsubsection{Įvykių atranka}

Šiame darbe buvo naudojami \ref{sec:data} skyrelyje aprašyti duomenys.
Kadangi $14$ TB su CMSSW dalinai apdorotų duomenų atsisiųsti iš Pietų Korėjos Tier3 duomenų centro yra sudėtinga ir nepatogu,
tai pradinis analizės etapas -- įvykių atranka -- buvo vykdoma nuotoliniu būdu prisijungus prie Tier3 duomenų centro.
Eksperimento metu užregistruoti bei modeliuoti protonų susidūrimo įvykiai buvo atrenkami taikant \ref{sec:selection}
skyrelyje aprašytus kriterijus.
Svarbiausia su atrinktais įvykiais susijusi informacija buvo išrašoma į naujus \ttt{.root} formato failus ir parsisiunčiama
į vietinį kompiuterį tolimesnei analizei.
Atrinktų duomenų failai bendrai užima apie $14$ GB, tad tokį kiekį duomenų gerokai patogiau parsisiųsti į asmeninį kompiuterį
ir jame saugoti.
Taip pat vėlesniuose analizės žingsniuose naudojant atrinktų duomenų failus gerokai sutrumpėja duomenų analizės laikas.
Palyginimui: įvykių atranka Tier3 duomenų centre, atrankos kodus leidžiant lygiagrečiai su keliolika skirtingų duomenų rinkinių
vienu metu, užtrunka apie $5$ valandas, o atsisiuntus duomenis tolimesnius analizės žingsnius iki grafikų gavimo galima gauti
per $2$-$3$ minutes.

\subsubsection{Eksperimento ir modeliavimo sąlygų nesutapimai}\label{sec:SFs}

Norint palyginti matavimą su modeliavimu svarbu yra ne tik sunormuoti modeliuotus įvykius pagal išmatuotą integruotąjį šviesį,
bet ir atsižvelgti į tai, kad pasitaiko nesutapimų tarp matavimo ir modeliavimo sąlygų, bei įvertinti jų įtaką atranką praeinančių
įvykių skaičiui, o taip pat ir kai kuriems įvykio metu užregistruotiems dydžiams.
Tai padaroma pritaikant įvairias pataisas.

Šiame darbe buvo naudojamos pataisos išmatuotoms elektronų energijos bei miuonų skersinio impulso vertėms.
Jos pritaikomos padauginant išmatuotąsias vertes iš pataisos daugiklio, kuris priklauso nuo pačios elektrono energijos ar miuono
skersinio impulso vertės.
Daugiklių vertes nustato tam skirtos CMS fizikinių objektų grupės (angl.\ \textit{Physics Object Group} -- POG).

Kitas neatitikimas tarp eksperimento ir modeliavimo yra protonų susidūrimų tankio pasiskirstymas (angliškai vadinamas
\textit{Pile-Up} -- PU).
Modeliuotuose duomenyse dažnai nesutampa įvykių, kuriuose dalelės išlėkė iš tam tikro pirminių viršūnių kiekio, skaičius.
Stengiantis į tai atsižvelgti modeliuotiems įvykiams priskiriami svoriai, kurie įvykius padaro labiau arba mažiau reikšmingais
pagal tai, koks protonų susidūrimų skaičius juose buvo sumodeliuotas.
Šiuos svorius pateikia už šviesį ir su juo susijusią informaciją atsakinga CMS fizikinių objektų grupė.

Taip pat šiame darbe buvo pritaikytos efektyvumų pataisos, įskaitančios trigerių suveikimo, leptonų atkūrimo, atpažinimo, bei miuonų
atskirumo įvertinimo efektyvumų neatitikimą eksperimentiniuose ir modeliuotuose duomenyse.
Kadangi visi šie elementai yra svarbūs įvykių atrankos procese, tai jie tiesiogiai nulemia atranką praieinančių įvykių skaičių.
Dėl šios priežasties pataisos, atsižvelgiančios į šiuos elementus yra pritaikomos kaip papildomi svoriniai daugikliai modeliuotiems
įvykiams.
Taigi, pasinaudojant \eqref{eq:NLOweight} formule galima užrašyti galutinę modeliuoto įvykio svorio išraišką:

\begin{equation}
	\omega_{i}^{\mathrm{Gal.}} = \omega_{i}^{\mathrm{Gen.}} \cdot
	\frac{ \sigma\Lumi }{ \sum_{j=1}^{N_{\mathrm{ev}}}\omega_{j}^{\mathrm{Gen.}} } \cdot
	\omega_{i}^{\mathrm{PU}} \cdot \omega_{i}^{\mathrm{Trig.}} \cdot \omega_{i}^{\mathrm{Atk.}} \cdot \omega_{i}^{\mathrm{Atp.}} \,
	( \cdot \, \omega_{i}^{\mathrm{Atsk.}} , \; \mathrm{jei \; tai \; miuonų \; įvykis} ).
	\label{eq:weightwSF}
\end{equation}
Čia $\omega_{i}^{\mathrm{Gal.}}$ -- galutinis modeliuotam įvykiui priskiriamas svoris, $\omega_{i}^{\mathrm{Gen.}}$ -- įvykiui
priskirtas svoris modeliavimo metu, $N_{\mathrm{ev}}$ -- atskirų modeliuotų įvykių skaičius duomenų rinkinyje, $\sigma$ --
duomenų rinkinio skerspjūvis, $\Lumi$ -- išmatuotas integruotasis šviesis, $\omega_{i}^{\mathrm{PU}}$ -- dėl protonų susidūrimo
tankio neatitikimo tarp matavimo ir modeliavimo įvykiui priskirtas svoris, $\omega_{i}^{\mathrm{Trig.}}$,
$\omega_{i}^{\mathrm{Atk.}}$,  $\omega_{i}^{\mathrm{Atp.}}$ -- atitinkamai svoriai dėl trigerio, leptono atkūrimo, bei leptono
atpažinimo efektyvumo pataisų, $\omega_{i}^{\mathrm{Atsk.}}$ -- svoris dėl miuono atskirumo įvertinimo efektyvumo pataisos.

\subsection{Drell-Yan proceso triukšmo įvykių skaičiaus įvertinimas $\emu$ metodu} \label{sec:emu}

Šiame darbe $\emu$ metodu buvo vertinamas triukšmo įvykių skaičius, susijęs su $\DYtau$, $\ttbar$, $tW$, $\tbarW$ ir $WW$
procesais tiek elektronų, tiek miuonų kanale.
Abiem atvejais kontrolinė sritis buvo apibrėžta viena -- elektrono ir miuono įvykiai, atrinkti pagal \ref{sec:selection}
skyrelyje aprašytus kriterijus.
Triukšmo įvykių skaičius kiekvienam leptonų poros invariantinės masės histogramos stulpeliui buvo skaičiuojamas naudojantis
\eqref{eq:emuReal} išraiška.
Nors tolimesnėje analizėje svarbus būtų tik bendras visų triukšmo įvykių skaičius, brėžiant grafikus yra visai naudinga
matyti kiekvieno proceso indėlį atskirai.
Norint galėti atvaizduoti kiekvieno proceso $\emu$ įvertį atskirai \eqref{eq:emuReal} galima šiek tiek pamodifikuoti.
Pavyzdžiui, $\ttbar$ proceso atveju galima $N_{\emu}^{\mathrm{Data}}$ narį padauginti iš santykio
$N_{\emu}^{\ttbar, \, \mathrm{MC}} / N_{\emu}^{Piln., \, \mathrm{MC}}$:

\begin{equation}
	N_{ll}^{\mathrm{Įv.}} =
	\frac{ N_{ll}^{\ttbar, \, \mathrm{MC}} }{ N_{\emu}^{\ttbar, \, \mathrm{MC}} }
	\cdot N_{\emu}^{\mathrm{Data}} \cdot
	\frac{N_{\emu}^{\ttbar, \, \mathrm{MC}}}{N_{\emu}^{\mathrm{Piln., \, MC}}}
	=
	\frac{ N_{ll}^{\ttbar, \, \mathrm{MC}} }{ N_{\emu}^{\mathrm{Piln., \, MC}} }
	\cdot N_{\emu}^{\mathrm{Data}} \; ,
	\label{eq:emuProc}
\end{equation}
čia $N_{\emu}^{\mathrm{Piln., \, MC}} = N_{\emu}^{\ttbar, \, \mathrm{MC}} + N_{\emu}^{tW, \, \mathrm{MC}} +
N_{\emu}^{\tbarW, \, \mathrm{MC}} + N_{\emu}^{\DYtau, \, \mathrm{MC}} + N_{\emu}^{WW, \, \mathrm{MC}}$  -- pilnutinis
modeliuotų $\emu$ įvykių skaičius.
Galime matyti, kad kokio proceso įvykių skaičių bevertintumėme, dalis išraiškos visada bus ta pati --  priklausomai nuo
proceso skirsis tik narys $N_{ll}^{\ttbar, \, \mathrm{MC}}$.

Norint, kad $\emu$ metodo įvertis būtų tikslesnis, reikia atsižvelgti ir į \ltq{netikrų} $\emu$ įvykių -- $\emu$ triukšmų
egzistavimą.
Tai yra tokie įvykiai, kuriuose čiurkšlė buvo klaidingai atpažinta kaip elekronas arba miuonas.
Šiame darbe tokių klaidingai kaip $\emu$ atpažintų įvykių skaičiaus įvertinimui buvo pasirinktas metodas, leidžiantis
įvertinti tokių triukšmų skaičių iš eksperimentinių duomenų, t.y.\ -- matavimu grįstas metodas.
Jį naudojant buvo daroma prielaida, kad visi $\emu$ triukšmo įvykiai yra $QCD$ įvykiai.
Šių įvykių skaičiaus įvertintimui buvo pasirinkta signalo sritis, kurioje užfiksuoti elektronas ir miuonas turi
priešingus elektrinius krūvius, ir kontrolinė sritis, kurioje užfiksuotų dalelių krūviai sutampa.
Triukšmo įvykių skaičius kontrolinėje srityje buvo apskaičiuojamas iš atranką praėjusių eksperimentinių įvykių skaičiaus
atėmus visų atranką praėjusių tikrų $\emu$ procesų skaičių, nustatytą iš modeliavimo.
Triukšmo įvykių skaičius kontrolinėje srityje buvo transformuojamas į skaičių signalo srityje pasinaudojant teoriškai
apskaičiuojamu daugikliu $1/R\approx 1/0.57$ \cite{AN2013}.
Gavus $\emu$ triukšmo įvykių skaičių $\emu$ signalo srityje atitinkamai buvo pataisomas eksperimentinių $\emu$
įvykių skaičius.
Tada $\emu$ metodo skaičiavimo formulė atrodo taip:

\begin{equation}
	N_{ll}^{\mathrm{Įv.}} = \frac{ N_{ll}^{\mathrm{Proc. \, MC}} }{ N_{\emu}^{\mathrm{Piln., \, MC}} }
	\cdot N_{\emu}^{\mathrm{Data} \, *} \; ,
	\label{eq:emuFinal}
\end{equation}
čia $N_{ll}^{\mathrm{Proc. \, MC}}$ bet kurio modeliuoto vertinamojo proceso dviejų vienodų leptonų galutinės būsenos
įvykių skaičius, $N_{\emu}^{\mathrm{Data} \, *} = N_{\emu}^{\mathrm{Data}} - N_{\emu}^{\QCD}$ --
patikslintas eksperimento metu užregistruotų $\emu$ įvykių skaičius, gautas iš pradinio skaičiaus atėmus iš matavimo
įvertintą $\emu$ triukšmo įvykių skaičių.

Vis dėlto, šiuo atveju padaryta prielaida, kad visi $\emu$ triukšmo įvykiai yra $QCD$ įvykiai, nėra visiškai teisinga.
Pakankamai svarų indėlį turi ir triukšmai, susiję su $\WJets$ procesu, todėl ateityje reikėtų pagalvoti apie tikslesnių
metodų $e\mu$ triukšmams įvertinti naudojimą.

%senas
\section{Drell-Yan proceso triukšmo įvykių skaičiaus įvertinimo rezultatai}

Šiame darbe buvo naudojami $ee$ (arba panašios į $ee$) ir $e\mu$ (arba panašios į $e\mu$) galutinių būsenų tikri 2015 metais CMS detektoriaus užfiksuoti protonų susidūrimo įvykiai, bei modeliuoti įvairių procesų įvykiai. Pagrindinis šio darbo tikslas buvo $e\mu$ metodu įvertinti Drell-Yan proceso triukšmo įvykių, siejamų su $\WW$, $tW$, $\bar{t}W$, $t\bar{t}$, $\DYtau$ procesais, skaičių, kad šį rezultatą sukombinavus su modeliavimu būtų galima tikroviškiau įvertinti bendrą $ee$ įvykių skaičių, nei naudojant vien modeliavimą.

\vspace{-0.35cm}
\subsection{Protonų susidūrimų įvykių atrankos rezultatai} \label{sec:ppResults}
\vspace{-0.2cm}

Modeliuotų įvykių normavimo procedūros aprašytos \ref{sec:MCweight} ir \ref{sec:PUreweight} skyriuose. Pagal integruotą šviesį įvykiai buvo pernormuoti priskiriant jiems skerspjūvius, išvardintus \ref{table:CrossSections} lentelėje. Išmatuoti integruoti šviesiai $ee$ ir $e\mu$ duomenų rinkiniuose buvo skirtingi: $2.26$ \invfb  $ee$ įvykiams ir $2.32$ \invfb  $e\mu$ įvykiams. \ref{table:CrossSections} lentelėje taip pat yra pateikiami tikėtini įvykių skaičiai esant $2.26$ \invfb integruotam šviesiui, bei atranką praėjusių modeliuotų dviejų leptonų įvykių skaičiai. Iš viso $ee$ įvykių atranką iš viso praėjo $934135$ tikrų ir $956579$ modeliuotų, o $e\mu$ -- $25817$ tikrų ir $26699$ modeliuotų įvykių, tad modeliavimas pervertina įvykių skaičių. Į su $Z$ bozono rezonansu siejamą invariantinės masės sritį ($60$ -- $120$ GeV) patenka $95.5\%$ tikrų ir $95.8\%$ modeliuotų $ee$ įvykių (labai didele dalimi Drell-Yan įvykiai), bei $44.8\%$ tikrų ir $48.08\%$ modeliuotų $e\mu$ įvykių. 

\vspace{-0.15cm}
\begin{centering}
\begin{table}[H]
\begin{tabular}{|c|c|c|c|c|}
\hline
\multirow{2}{*}{Procesas} & Reakcijos & Tikėtinas įvykių  & Atranką praėjusių & Atranką praėjusių\\
  &  skerspjūvis (pb) & skaičius & $ee$ įvykių skaičius & $e\mu$ įvykių skaičius \\
\hline\hline
$\mathrm{DY} \! \rightarrow \! e^{+}e^{-}$ & $8211.73$ & $1.85585\cdot 10^{7}$ & $939228$ & -- \\
\hline
$\mathrm{DY} \! \rightarrow \! \tau^{+} \tau^{-}$ & $8211.73$ & $1.85585\cdot 10^{7}$ & $2636$ & $5720$ \\
\hline
$t\bar{t}$ & $831.76$ & $1.87978\cdot 10^{6}$ & $7137$ & $17260$ \\
\hline
$\dtW$ & $76.18$ & $1.72167\cdot 10^{5}$ & $800$ & $1792$ \\
\hline
$\WJets$ & $61526$ & $1.39049\cdot 10^{8}$ & $1319$ & $985$ \\
\hline
$\WW$ & $118.7$ & $2.68262\cdot 10^{5}$ & $874$ & $1927$ \\
\hline
$\WZ$ & $47.13$ & $1.06514\cdot 10^{5}$ & $975$ & $187$ \\
\hline
$\ZZ$ & $16.523$ & $3.7432\cdot 10^{4}$ & $675$ & $24$ \\
\hline
$\gJets$ & $3.65896\cdot 10^{5}$ & $8.26925\cdot 10^{8}$ & $86$ & $34$ \\
\hline
$\QCD$ & $5.19096\cdot 10^{6}$ & $1.173157\cdot 10^{10}$ & $2847$ & $490$ \\
\hline
\end{tabular}
\caption{Drell-Yan proceso elektronų kanalo signalas ir pagrindiniai triukšmai, modeliavime naudoti jų reakcijų skerspjūviai, kai protonų susidūrimo energija $\sqrt{s}=13$ TeV, tikėtinas įvykių skaičius, kai protonų susidūrimų integruotas šviesis yra $\Lumi=2.26$~\invfb, bei atranką praėjusių modeliuotų $ee$ ir $e\mu$ įvykių skaičius.}
\label{table:CrossSections}
\end{table}
\end{centering}
\vspace{-0.15cm}

Pasinaudojant dalelių srauto algoritmu apskaičiuojamu miuonų atskirumu buvo sumažintas $e\mu$ įvykių atranką praeinančių $\WJets$ triukšmo įvykių skaičius. Rezultatas atsispindi \ref{fig:emuAfter} paveiksle, kuriame $\WJets$ įvykių skaičius vaizduojamas violetiniuose stulpeliuose. Pritaikius apribojimus atskirumui, apskaičiuojamam naudojantis dalelių srauto algoritmu, šių stulpelių aukštis gerokai sumažėjo.

Gautos modeliuotų ir tikrų $ee$ įvykių invariantinės masės histogramos vaizduojamos \ref{fig:eeInvM} pav., o modeliuotų ir tikrų $e\mu$ įvykių -- \ref{fig:emuInvM} pav.

Pats įvykių atrankos vykdymo procesas užėmė daugiausiai šio darbo laiko. Norint atrinkti visus tikrus ir modeliuotus įvykius su įprastu kompiuteriu užtrunka apie $4$ valandas. Ateityje būtų tikslinga įvairiais būdais pabandyti šį laiką sutrumpinti, pavyzdžiui, efektyvinant kodus, skaičiuojant lygiagrečiai ir pan. Taip pat, atranką galima būtų vykdyti keliais etapais. Pavyzdžiui, vieną kartą įvykių atranką atlikti tik remiantis trigerio aktyvavimu ir užfiksuotų dalelių skaičiumi (ir tokią atranką praėjusius įvykius įrašant į atskirą failą). Tada tolimesnei atrankai, kuriai norime išbandyti įvairesnius kriterijus, naudoti tik pirminę atranką praėjusius įvykius.

%\vspace{-1cm}
\subsubsection{Didelio tikėtinumo įvykiai}
%\vspace{-0.2cm}

\ref{fig:eeInvM}-\ref{fig:emuInvM} pav.\ galima pastebėti, kad kai kurių procesų modeliuotų įvykių skaičiaus pasiskirstymas yra labai netolygus (matosi pavieniai aukšti histogramos stulpeliai). Taip yra todėl, kad tokie įvykiai, kaip $\QCD$ ir $\gJets$ yra labai didelio tikėtinumo (tai matosi \ref{table:CrossSections} lentelėje -- šių procesų tikėtinumas, lyginant su Drell-Yan proceso tikėtinumu, didesnis smarkiau, nei eile). Modeliuojant didelio tikėtinumo įvykius jiems priskiriami labai dideli svoriai. Didelių svorių priskyrimas įvykiams turi savo kainą. Taikant atrankos kriterijus, kurie labai smarkiai sumažina tokių įvykių atrankos praėjimo tikimybę, vistiek atsiranda bent keli įvykiai, kurie sugeba ją praeiti. Tačiau, kadangi tokių įvykių yra labai mažai, o jų svoriai labai dideli, histogramose matome tik keletą aukštų stulpelių. Taip pat, iš \eqref{eq:w2sumUnc} išraiškos matyti, kad didžiausią statistinę paklaidą įneša įvykiai, kurie turi labai didelius svorius. Taigi beveik visą modeliuotų įvykių skaičiaus statistinę paklaidą sudaro $\QCD$ įvykiai (kurie dar ją ir labai išpučia). $\QCD$ įvykių paklaida beveik lygi pačiam įvykių skaičiui (pavyzdžiui, $ee$ atranką praėjo $2847$ įvykiai, o vien $\QCD$ įvykių skaičiaus statistinė paklaida lygi $2797$).

Norint pamatyti realistiškesnį (tolygesnį) tokių įvykių pasiskirstymą bei gauti mažesnes statistines paklaidas, reikėtų turėti labai didelį skaičių įvykių, kad jiems būtų galima priskirti artimus vienetui svorius (pavyzdžiui, šiame darbe naudoti $\QCD$ įvykiai turi svorius, lygius keletui dešimčių ar net šimtų įvykių). Vis dėlto, labai didelis įvykių skaičius nėra patogus laiko ir kompiuterio atminties atžvilgiu, nes bet kuriuo atveju beveik visi tokie įvykiai atrankos metu yra \ltq{išmetami}.

%\vspace{-0.5cm}
\subsection{$e\mu$ metodo rezultatai}
%\vspace{-0.2cm}

\ref{sec:ppResults} skyrelyje buvo paminėta, kad modeliavimas pervertina įvykių skaičių, taigi būtų galima tikėtis, kad, įvertinus įvykių skaičių $e\mu$ metodu, jis turėtų būti sumažintas. Kaip buvo rašoma \ref{sec:emu} skyrelyje, $e\mu$ metodo skaičiavimus buvo bandoma patikslinti įskaitant $\WJets$ ir kitų $e\mu$ triukšmo įvykių įtaką. \ref{table:emu} lentelėje matyti, kad, įvykių skaičius labiausiai sumažinamas, kai įskaitomi visi $e\mu$ triukšmo įvykiai (įvykių skaičiaus sumažėjimas, kai įskaitomi visi triukšmai, ir kai neįskaitomi išvis, skiriasi $2.26$ karto). Vis dėlto, kadangi $\WJets$ yra dominuojantis $e\mu$ triukšmo įvykis, įvykių skaičiaus sumažinimas, kai įskaitoma vien $\WJets$ įtaka, ir kai įskaitomi visi triukšmo procesai, skiriasi tik $15.8\%$. $e\mu$ metodo įverčio, kai atsižvelgiama į visus $e\mu$ triukšmo procesus, palyginimas su modeliavimu pateikiamas \ref{fig:EMuEst} pav. Rezultatas, gaunamas vietoje modeliuotų įvykių įstačius $e\mu$ metodo įverčius, pateikiamas \ref{fig:eeInvMest} pav.
%Matavimo/įverčio santykiai, gauti, kai buvo naudojamas tik modeliuotas įvertis (\ref{fig:eeInvM} pav.), ir kai buvo naudojamas kombinuotas $e\mu$ metodo bei modeliuotas įverčiai (\ref{fig:eeInvMest} pav.), viename grafike yra pavaizduoti \ref{fig:estChange} pav., kad būtų galima pastebėti santykio pasikeitimą dėl $e\mu$ metodo naudojimo.

\begin{centering}
\begin{table}[H]
\begin{tabular}{|c|c|}
\hline
Triukšmų įskaitymas & $e\mu$ metodu įvertintas įvykių skaičius \\
\hline \hline
Triukšmų įtaka neįskaitoma & $4881\pm 64$ \\
\hline
Įskaitomas tik $\WJets$ indėlis & $4715\pm 69$ \\
\hline
Įskaitomi visi $e\mu$ triukšmo procesai & \multirow{2}{5em}{\centering $4668\pm 69$} \\
($\WJets$, $\gJets$, $\QCD$, $\WZ$, $\ZZ$) & \\
\hline
\end{tabular}
\vspace{-0.2cm}
\caption{\label{table:emu} $e\mu$ metodu įvertintas Drell-Yan proceso triukšmo įvykių skaičius su $Z$ bozono rezonansu siejamoje srityje, kai atsižvelgiama į skirtingą $e\mu$ triukšmo procesų skaičių. Nurodytos tik statistinės paklaidos.}
\end{table}
\end{centering}

\begin{centering}
\begin{table}[H]
\begin{tabular}{|c|c|c|c|} %|p{5cm}|p{4cm}|p{8cm}|p{9cm}|
\hline 
\multirow{3}{8em}{\centering Įvykiai} & \multirow{3}{7em}{\centering CMS užfiksuotų įvykių skaičius} & \multirow{3}{9em}{\centering Modeliuotų įvykių skaičius} & \multirow{3}{10em}{\centering Modeliuotų įvykių skaičius taikant $e\mu$ metodo įvertį} \\
 & & & \\
 & & & \\
\hline \hline
\multirow{2}{8em}{\centering $e\mu$} & \multirow{2}{7em}{\centering $25817 \pm 161$} & \multirow{2}{9em}{\centering $26699 \pm 86 \pm 246$ } & \multirow{2}{5em}{\centering \textendash }\\
 & & & \\
\hline
$ee$ ($\WW$, $tW$, $\bar{t}W$, & \multirow{2}{7em}{\centering\textendash} & \multirow{2}{9em}{\centering $11448 \pm 59 \pm 475$} & \multirow{2}{9em}{\centering$\mathbf{10613} \pm 107 \pm 469$} \\
$t\bar{t}$, $\DYtau$) & & & \\
\hline
\multirow{2}{8em}{\centering $ee$ (visi procesai)} & \multirow{2}{7em}{\centering $934135\pm 967$} & \multirow{2}{10em}{\centering $956579\pm 2971\pm11632$} & \multirow{2}{10em}{\centering $\mathbf{955745} \pm 2971 \pm 11656$} \\
 & & & \\
\hline
\multirow{2}{8em}{\centering $N_{ee}/N_{ee}^{\mathrm{Obs.}}$} & \multirow{2}{7em}{\centering 1} & \multirow{2}{10em}{\centering $1.024 \pm 0.003 \pm 0.012$} & \multirow{2}{10em}{\centering $1.023 \pm 0.003 \pm 0.012$} \\
 & & & \\
\hline
\end{tabular}
\caption{\label{table:finalResults} CMS detektoriumi užregistruotų dviejų leptonų įvykių skaičius tirtame dviejų leptonų invariantinės masės intervale. Jeigu prie rezultato pateikiamos dvi paklaidos, tai pirmoji iš jų yra statistinė, o antroji -- sisteminė paklaida.}
\end{table}
\end{centering}

\vspace{-0.3cm}
\begin{centering}
\begin{table}[H]
\begin{tabular}{|c|c|c|c|} %|p{5cm}|p{4cm}|p{8cm}|p{9cm}|
\hline 
\multirow{3}{8em}{\centering Įvykiai} & \multirow{3}{7em}{\centering CMS užfiksuotų įvykių skaičius} & \multirow{3}{9em}{\centering Modeliuotų įvykių skaičius} & \multirow{3}{10em}{\centering Modeliuotų įvykių skaičius taikant $e\mu$ metodo įvertį} \\
 & & & \\
 & & & \\
\hline \hline
\multirow{2}{8em}{\centering $e\mu$} & \multirow{2}{7em}{\centering $11535 \pm 107$} & \multirow{2}{9em}{\centering $12838 \pm 315 \pm 180$ } & \multirow{2}{5em}{\centering \textendash }\\
 & & & \\
\hline
$ee$ ($\WW$, $tW$, $\bar{t}W$, & \multirow{2}{7em}{\centering\textendash} & \multirow{2}{9em}{\centering $5013 \pm 38 \pm 173$} & \multirow{2}{9em}{\centering$\mathbf{4668} \pm 69 \pm 209$} \\
$t\bar{t}$, $\DYtau$) & & & \\
\hline
\multirow{2}{8em}{\centering $ee$ (visi procesai)} & \multirow{2}{7em}{\centering $894660\pm 946$} & \multirow{2}{10em}{\centering $913943 \pm 972 \pm 11201$} & \multirow{2}{10em}{\centering $\mathbf{913598} \pm 973 \pm 11233$} \\
 & & & \\
\hline
\multirow{2}{8em}{\centering $N_{ee}/N_{ee}^{\mathrm{Obs.}}$} & \multirow{2}{7em}{\centering 1} & \multirow{2}{10em}{\centering $1.022 \pm 0.002 \pm 0.013$} & \multirow{2}{10em}{\centering $1.021 \pm 0.002 \pm 0.013$} \\
 & & & \\
\hline
\end{tabular}
\caption{\label{table:finalResultsZ} CMS detektoriumi užregistruotų ir modeliuotų dviejų leptonų įvykių skaičius $Z$ rezonanso aplinkoje ($60$ -- $120$ GeV) Jeigu prie rezultato pateikiamos dvi paklaidos, tai pirmoji iš jų yra statistinė, o antroji -- sisteminė paklaida.}
\end{table}
\end{centering}

%\vspace{-0.3cm}
Nors triukšmo įvykių skaičius invariantinės masės histogramos stulpeliuose iki $700$ GeV yra sumažinamas iki $20\%$, tačiau lyginant įvykių skaičiaus pasikeitimą su visų $ee$ įvykių skaičiumi, šis sieka $0.1\%$. Taip yra todėl, kad bendras visų įvykių skaičius yra labai didelis, o triukšmų skaičius ganėtinai mažas (dominuoja Drell-Yan procesas), todėl bendrame kontekste toks triukšmo įvykių skaičiaus pasikeitimas sunkiai pastebimas, be to, yra nemažas skaičius triukšmo įvykių (išvardinti \ref{table:CrossSections} lentelėje), kurių $e\mu$ metodu įvertinti negalima.
%Į laisvės laipsnių skaičių (invariantinės masės histogramos stulpelių skaičių) normuotas sutapimą tarp matavimo ir (modeliuoto arba $e\mu$ metodo) įverčio apibūdinantis dydis $\chi^{2}/n_{\mathrm{l. \, l.}}$ buvo skaičiuojamas pagal tokią formulę:
%\begin{equation}
%\chi^{2}/n_{\mathrm{l. \, l.}}=\frac{1}{n_{\mathrm{l. \, l.}}} \sum_{i=1}^{n_{\mathrm{l. \, l.}}} \frac{ \left( \rho_{i}^{\mathrm{Įv.}}-\rho_{i}^{\mathrm{Data}} \right) ^{2} }{\rho_{i}^{\mathrm{Data}}} \; \mathrm{,}
%\label{eq:chiSquare}
%\end{equation}
%čia $\rho_{i}^{\mathrm{Įv.}}$ -- modeliuotas ($\rho_{i}^{\mathrm{MC}}$), arba kombinuotas $e\mu$ metodo bei modeliuotas ($\rho_{i}^{\mathrm{MC}+e\mu \, \mathrm{įv.}}$) įvykių skaičiaus tankio įvertis histogramos stulpelyje. Įvykių skaičiaus tankis skaičiuojamas padalinus įvykių skaičių iš stulpelio pločio.
%:
%\begin{equation}
%\rho_{i}=\frac{N_{i}}{m_{\, i \, 1}-m_{\, i \, 0}} \; \mathrm{,}
%\label{eq:numDens}
%\end{equation}
%čia $m_{\, i \, 0}$ ir $m_{\, i \, 1}$ -- $i$-ojo histogramos stulpelio kairysis ir dešinysis rėžiai.
%Iš \eqref{eq:chiSquare} išraiškos matyti, kad kuo $\chi^{2}/n_{\mathrm{l. \, l.}}$ vertė mažesnė, tuo geresnis sutapimas tarp matavimo ir įverčio. Šio dydžio vertės buvo gautos tokios: $\chi^{2}_{\mathrm{MC}}/n_{\mathrm{l. \, l.}}=22.44$; $\chi^{2}_{\mathrm{MC}+e\mu \, \mathrm{įv.}}/n_{\mathrm{l. \, l.}}=22.31$ -- kombinuojant modeliuotą ir $e\mu$ metodo įverčius gaunamas maždaug $0.6\%$ geresnis sutapimas, nei naudojant vien tik modeliavimą.

Gauta $e\mu$ metodo įverčio statistinė paklaida yra $1.82$ karto didesnė už modeliuotų įvykių skaičiaus paklaidą\footnote{Čia nagrinėjami rezultatai, gauti su $Z$ bozono rezonansu siejamoje srityje ($60$ -- $120$ GeV).}. Taip yra todėl, kad įvertis gautas iš kelių duomenų rinkinių, kurių visų statistinės paklaidos įeina į įverčio paklaidą pagal \eqref{eq:DerUnc} formulę. Vis dėlto, statistines paklaidas užgožia apskaičiuotos sisteminės paklaidos. $e\mu$ metodo įverčio sisteminė paklaida už modeliuotų įvykių skaičiaus sisteminę paklaidą yra $21\%$ didesnė. Bendrame kontekste statistinė įverčio paklaida yra maža, lyginant su pilna statistine įvykių skaičiaus paklaida (vien $\QCD$ įvykių skaičiaus statistinė paklaida yra apie $13$ kartų didesnė už $e\mu$ metodo įverčio statistinę paklaidą), taip pat, kaip ir sisteminė paklaida, kuri yra labai maža lyginant su bendra sistemine paklaida, priklausančia nuo trigerių pasirinkimo ir normavimo pagal protonų susidūrimų tankį (bendros sisteminės paklaidos dydis yra lygus $11233$, ji apie $54$ kartus didesnė už $e\mu$ įverčio paklaidą). Taigi bendrame kontekste skirtumas tarp modeliavimo ir $e\mu$ metodo rezultato telpa į paklaidas. Kita vertus, jeigu žiūrime tik į tuos procesus, kurių įvykių skaičių buvo galima įvertinti $e\mu$ metodu, tada modeliavimo ir $e\mu$ įverčio rezultatai yra vienas už kito paklaidų ribų (pačios paklaidos persidengia 52 įvykių intervale), tad galime teigti, kad įvykių skaičiaus įvertintimas $e\mu$ metodu buvo tikslingas.

Tyrimų rezultatai yra apibendrinti \ref{table:finalResults} ir \ref{table:finalResultsZ} lentelėse. Pateikti skaičiai atitinka matavimą detektoriaus geometrinėje erdvėje, kai greitesniojo elektrono $\pT^{\mathrm{Gr.}}>30$ GeV, o lėtesniojo $\pT^{\mathrm{Lėt}}>10$ GeV ($ee$ įvykiams), arba kai elektrono $\pT^{e}>25$ GeV, o miuono $\pT^{\mu}>15$ GeV ($e\mu$ įvykiams).

Norint labiau patikslinti visų procesų įvykių skaičiaus įvertį, reikėtų $e\mu$ metodui netinkamų procesų įvykių skaičių įvertinti kokiu nors kitu matavimu grįstu (pavyzdžiui, klaidingo atpažinimo) metodu, bei patikslinti modeliuotų Drell-Yan signalo įvykių skaičių. Taip pat ateityje sisteminių paklaidų skaičiavime būtų galima pabandyti įtraukti ir daugiau galimų sisteminių nuokrypių šaltinių.

Darbe buvo atsižvelgta į šiuos sisteminių paklaidų šaltinius:
\begin{enumerate}
%\vspace{-0.25cm}
\item Modeliuotų įvykių skaičiaus pokytį dėl pernormavimo pagal protonų susidūrimų tankį;
%\vspace{-0.3cm}
\item Modeliuotų įvykių skaičiaus nesutapimą, kai įvykių atrankai naudojami skirtingi trigeriai;
%\vspace{-0.95cm}
\item Galimai įneštus sisteminius nukrypimus dėl $e\mu$ metodo naudojimo.
\end{enumerate}

%\vspace{-0.15cm}
Parašyti vykdytos duomenų analizės programiniai kodai su minimaliais pataisymais gali būti tinkami naudojimui ir su naujesnių metų CMS detektoriaus duomenimis. Tai buvo išbandyta naudojant šiek tiek skirtingus $2015$ metų duomenų rinkinius.

\clearpage
\section*{Išvados} \addcontentsline{toc}{section}{Išvados}
\begin{enumerate}
\item Didelių energijų fizikos duomenų analizės darbas reikalauja daug laiko ir kruopštumo.
\item Įvertinus Drell-Yan proceso triukšmo įvykių skaičių $e\mu$ metodu, gautas pilnutinis įvykių skaičius buvo artimesnis užregistruotajam CMS detektoriumi, nei naudojant vien tik modeliavimo įvertį ($835$ įvykiais, arba apie $0.1\%$).
\item Vertinant tik tų procesų, kuriems galima taikyti $e\mu$ metodą, įverčius, $e\mu$ ir modeliavimo įverčiai paklaidų ribose nėra tapatūs (modeliavimo įvertis -- $5013\pm38\pm173$, $e\mu$ metodo įvertis -- $4668\pm69\pm209$), tad $e\mu$ metodo taikymas yra tikslingas.
\item $e\mu$ metodo skaičiavimuose atsižvelgus į panašių į $e\mu$ procesų (\ltq{$e\mu$ triukšmų}: $\WJets$ $\gJets$, $\QCD$, $\WZ$, $\ZZ$), kurie iškraipo $e\mu$ įvykių skaičių, egzistavimą, galima pagerinti įvertį.
\item $e\mu$ metodo tikslumas, lyginant su modeliuotu įverčiu, yra žemesnis ($e\mu$ metodu įvertinto įvykių skaičiaus santykinė paklaida -- $4.7\%$, modeliuoto įverčio santykinė paklaida -- $3.6\%$), tačiau gautas įvykių skaičius yra patikimesnis, nes modeliavime yra nežinomų (ir todėl neįskaitomų) neapibrėžtumų.
\item Labai didelio tikėtinumo triukšmo įvykiai su maža tikimybe praeina atrankos kriterijus, todėl jų pasiskirstymą sunku pakankamai tiksliai modeliuoti. Tai nulemia \ltq{pikų} atsiradimą histogramose.
\item Pakeitus miuonų atrankos reikalavimus (vietoje trekerio atskirumo naudojant atskirumą, apskaičiuojamą naudojantis dalelių srauto algoritmu) galima sumažinti $e\mu$ atranką praeinančių $\WJets$ įvykių skaičių.
\item Parašyti protonų susidūrimo įvykių atrankos, $e\mu$ metodo skaičiavimų bei grafikų brėžimo programiniai kodai su minimaliomis korekcijomis galės būti taikomi ir atliekant naujesnių metų CMS duomenų analizę.
\end{enumerate}

\clearpage
%\addtocounter{section}{1}
%\addcontentsline{toc}{section}{\protect\numberline{\thesection}Literatūra}
\addcontentsline{toc}{section}{Literatūros sąrašas}
\bibliography{\jobname}
\bibliographystyle{unsrt}

\clearpage
\section*{Santrauka}
\addcontentsline{toc}{section}{Santrauka (LT)}
\begin{centering}
Marijus Ambrozas\\
\textbf{Drell-Yan proceso triukšmo įvykių skaičiaus įvertinimas $e\mu$ metodu}\\
\end{centering}
\vspace{0.5cm}

Preciziškas Drell-Yan proceso eksperimentinis tyrimas leidžia tikslinti partonų pasiskirstymo funkcijas (angl.\ \textit{parton distribution functions} -- PDF), taip pat įvairius teorinius modelius ir pataisas. Norint gauti kuo tikslesnius Drell-Yan proceso eksperimentinius duomenis, svarbu yra įvertinti Drell-Yan triukšmo įvykių skaičių. Kad būtų sumažinta su matavimu nesusijusių neapibrėžtumų (pvz., nepakankamai tiksliai žinomų įvykių skerspjūvių, detektoriaus atsako modeliavimo netobulumų) įtaka, triukšmo įvykių skaičiaus įvertinimui yra naudojami matavimu grįsti metodai. Šio darbo tikslas buvo $e\mu$ metodu (vienas iš matavimu grįstų metodų) įvertinti $\WW$, $tW$, $\bar{t}W$, $t\bar{t}$ ir $\DYtau$ procesų nulemtų Drell-Yan proceso elektronų kanalo triukšmo įvykių skaičių. Darbe buvo naudojami $2015$ metų CERN CMS detektoriaus duomenys. Kad būtų galima pritaikyti $e\mu$ metodą, visų pirma reikėjo įvykdyti eksperimentinių ir modeliuotų elektrono-pozitrono ($ee$) poros galutinės būsenos bei elektrono ir miuono ($e\mu$) galutinės būsenos įvykių atranką. Modeliuoti įvykiai turėjo būti pernormuoti, kad atitiktų išmatuotąjį integruotą šviesį bei protonų susidūrimų tankį. Tada pasinaudojant $e\mu$ metodu iš CMS detektoriumi registruotų $e\mu$ duomenų, bei modeliuotų $ee$ ir $e\mu$ duomenų buvo apskaičiuotas išvardintų procesų $ee$ galutinės būsenos įvykių skaičiaus įvertis. Vietoje modeliuotų duomenų panaudojus $e\mu$ metodo įvertį gautas rezultatas buvo šiek tiek artimesnis eksperimentiniam rezultatui (minėtų procesų $ee$ galutinės būsenos įvykių skaičius buvo žymiai sumažintas, tačiau, kadangi tai nėra vieninteliai Drell-Yan proceso triukšmai, bendrame kontekste toks sumažėjimas nėra labai aiškiai pastebimas). Taip pat šiame darbe buvo įvertinta įvykių skaičiaus sisteminė paklaida, nulemta trijų skirtingų šaltinių. Darbo metu parašyti programiniai kodai su minimaliomis korekcijomis bus tinkami naudoti su bet kurių metų CMS detektoriaus duomenimis.

\clearpage
\section*{Summary}
\addcontentsline{toc}{section}{Santrauka (EN)}
\begin{centering}
Marijus Ambrozas\\
\textbf{Drell-Yan Process Background Estimation Using $e\mu$ Method}\\
\end{centering}
\vspace{0.5cm}
The precise experimental analysis of the Drell-Yan process allows constraining parton distribution functions (PDFs), as well as testing various theoretical models and corrections. Contribution of background events must be considered in order to perform Drell-Yan measurements as precisely as possible. Some uncertainties in the measurement originate from the simulations of distinct processes, others are related to the response of the detector. These uncertainties can be reduced using data-driven methods. The goal of this work was to estimate the number of Drell-Yan background events originating from $\WW$, $tW$, $\bar{t}W$, $t\bar{t}$ and $\DYtau$ processes using the $e\mu$ method (one of the data-driven methods). The work was performed using CMS detector's data from $2015$. Before using the $e\mu$ method, the electron-positron ($ee$) final state as well as electron-muon ($e\mu$) final state event selection had to be done for the data and MC datasets. The MC events had to be reweighted to match the observed integrated luminosity and the measured distribution of reconstructed vertices. The $e\mu$ method was allowed estimation of the $ee$ final state background events by combining the number of real $e\mu$ data events, MC $e\mu$ events and MC $ee$ events. The final result was obtained by replacing the number of $\WW$, $tW$, $\bar{t}W$, $t\bar{t}$ and $\DYtau$ MC events with the estimated number of Drell-Yan background events. The resultant number of events was a little bit closer to the measured number of events (the number of previously mentioned background events was significantly reduced, but due to the fact that there are more different background processes that could not be estimated using the $e\mu$ method, the reduction of background events did not seem that strong when compared to the full number of events). The systematic uncertainties from three different sources were estimated. The C++ scripts written for this work will be suitable for performing the same analysis with more recent CMS data.

\clearpage
\section*{Bibliografinis aprašas}
%\addcontentsline{toc}{section}{Bibliografinis aprašas}
Marijus Ambrozas. Drell-Yan proceso triukšmo įvykių skaičiaus įvertinimas $e\mu$ metodu. Fizikos bakalauro studijų programos baigiamasis darbas. Vad.\ Andrius Juodagalvis. Vilnius: Vilniaus universitetas, Fizikos fakultetas, 2018.

%\section*{Anotacija}
%\addcontentsline{toc}{section}{Anotacija}
%Eksperimentiniame Drell-Yan proceso tyrime svarbus žingsnis yra triukšmų įvertinimas. Šiame darbe dalis Drell-Yan proceso triukšmo įvykių buvo įvertinta $e\mu$ metodu. Pritaikius šį metodą bendras įvykių skaičiaus įvertis buvo šiek tiek priartintas prie išmatuoto įvykių skaičiaus. Darbe buvo įvertintos skaičiavimų statistinės ir sisteminės paklaidos. Darbas buvo atliekamas rašant programinius kodus ir naudojant 2015 metų CMS detektoriaus duomenis. Parašytus kodus su minimaliomis korekcijomis bus galima naudoti ir su vėlesnių metų CMS detektoriaus duomenimis. 
\end{document}