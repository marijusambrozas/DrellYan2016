\documentclass[a4paper, 12pt]{article}
\usepackage[yyyymmdd]{datetime}
\usepackage{fontspec}
\usepackage{fontenc}
\usepackage{amssymb}
\usepackage{ulem}
\usepackage{cite}
\usepackage{multirow}
\usepackage{mathtools}
\usepackage{amsmath}
\usepackage{float}
\usepackage{graphicx}
%\usepackage{url}
\usepackage{hyperref}
\usepackage{caption}
\usepackage[svgnames]{xcolor}
\usepackage{xstring} % to define myund
\usepackage[lithuanian]{babel}
%\usepackage{lineno}
\graphicspath{ {Images/} }

\renewcommand{\dateseparator}{-}
\renewcommand{\abstractname}{Santrauka}
\renewcommand{\contentsname}{Turinys}
\renewcommand{\figurename}{Pav}
\renewcommand{\refname}{Literatūra}
\renewcommand{\tablename}{Lentelė}
\renewcommand{\listfigurename}{Paveikslėlių sąrašas}
\renewcommand{\listtablename}{Lentelių sąrašas}

\DeclareCaptionLabelFormat{numfirst}{#2~#1}
%\captionsetup[figure]{labelformat = numfirst, labelsep = period}
\captionsetup[figure]{labelformat = numfirst, labelsep = space}
\captionsetup[table]{labelformat = numfirst, labelsep = period}

\newcommand{\textblue}[1]{{\color{Blue}#1}}
\newcommand{\textred}[1]{{\color{Red}#1}}
\newcommand{\comment}[1]{\newline\textblue{#1}\newline}
\newcommand{\commentNL}[1]{\textblue{#1}\newline}
\newcommand{\commentMA}[1]{\newline\textred{#1}\newline}
\newcommand{\commentMANL}[1]{\textred{#1}\newline}
\newcommand{\pT}{p_{T}}
\newcommand{\DYtau}{\mathrm{DY}\!\rightarrow\tau\tau}
\newcommand{\DYee}{\mathrm{DY}\!\rightarrow ee}
\newcommand{\QCD}{\mathrm{QCD}}
\newcommand{\Data}{\mathrm{Data}}
\newcommand{\MC}{\mathrm{MC}}
\newcommand{\est}{\mathrm{est.}}
\newcommand{\emu}{e\mu}
\newcommand{\ET}{E_{T}}
\newcommand{\refeqq}[1]{(\ref{#1})}
\newcommand{\Lumi}{{\cal L}_\mathrm{int}}
\newcommand{\invfb}{fb$^{-1}$}
\newcommand{\invpb}{pb$^{-1}$}
\newcommand{\ltq}[1]{{\quotedblbase{}#1\textquotedblleft{}}}
\newcommand{\beq}{\begin{equation}}
\newcommand{\eeq}{\end{equation}}
\newcommand{\ttt}[1]{\texttt{#1}}

\def\myund#1{%
  \saveexpandmode\expandarg
  \IfSubStr{#1}{_}{%
   \StrSubstitute{#1}{_}{\_}}{#1}%
  \restoreexpandmode
}


\begin{document}

%\linenumbers

\title{Darbų dienoraštis}
\author{M.\ Ambrozas}
\date{\textit{Pradėta} 2018-10-09, \textit{Paskutinė versija} \today}
\maketitle
\pagebreak

\section{Spalio 9-10 d. -- CMSDAS ataskaitos taisymas}
\begin{itemize}
	\item Tikrinau, ar veikia neinteraktyvus skaičiavimas tier3 centre
	(prieš tai kelis kartus buvo išmetę klaidą, kad neįmanoma įrašyti atrinktų
	įvykių į failą, nes nėra tinkamo serverio, tačiau, kai pabandžiau paleisti
	tuos pačius skaičiavimus, tik su 100 įvykių kiekvienam procesui, viskas veikė
	normaliai) -- pabandžiau paleisti atskirus skaičiavimus su $WW$, $WZ$, $ZZ$,
	$tW$, $\overline{t}W$ procesais. Panašu, kad viskas suveikė normaliai, failų
	įrašymas buvo sėkmingas. Dar ko gero reikėtų pabandyti paleisti skaičiavimą
	su kitais procesais, kad galima būtų spręsti, ar ten buvo tik laikina problema
	su tier3 sistemomis, ar ši problema priklauso nuo bandomų įrašyti failų dydžio,
	ar nuo sistemos vykdomų skirtingų darbų skaičiaus ir pan.
	\item Trumpai peržiūrėjau \textit{DY working meeting 2018-06-22} skaidres
	(\url{https://indico.cern.ch/event/736672/contributions/3038893/
	attachments/1673244/2685121/DY_working_meeting_20180622.pdf}).
	Panašu, kad ten buvo naudojamas vienodas integruotas šviesis tiek $ee$, tiek
	$\mu\mu$ įvykiams, o pritaikius korekcijas (PU, Rochester (miuonams),
	track (miuonams), reco (elektronams) ID, iso (miuonams), trigger) nesutapimai
	tarp eksperimento ir modeliavimo siekia maždaug iki 5 procentų. Kadangi mano
	atveju taip nėra, trupmai užmečiau akį i mano korekcijų kodus (kuriuos paėmęs
	iš Kyeongpil Lee github saugyklos netaisiau) -- bent jau prie elektronams
	taikomų korekcijų nėra bandoma įvertinti trigerio efektyvumo. Taip pat skaidrėse
	vaizduojami PU pasiskirstymai neatrodo panašūs į mano turimus (skaidrėse
	esantys pasiskirstymai atrodo panašūs į 2015 metų duomenų pasiskirstymą, tik
	PU vertės vidurkis didesnis, o mano turimi pasiskirstymai turi kažką panašaus
	į lokalų maksimumą ties mažesnėmis PU vertėmis). Taip pat ir po PU korekcijos
	vaizduojama PU pasiskirstymo Data/MC kreivė savo forma atrodo labai nepanaši
	į mano (nors taip galbūt gali būti ir todėl, kad nežinau, ar savo grafikuose
	atvaizduoju \ltq{tikrą} eksperimentinį PU pasiskirstymą: jį pavaizduoju pagal
	failą, kuriuo naudojantis skaičiuojama modeliuotų įvykių PU svorio vertė, nes
	eksperimentiniuose dydžiuose kintamasis \ttt{nPileUp} visur duoda 0.
	\textbf{Reikia pasidomėti, kaip daromos tos korekcijos ir pabandyti pasitaisyti
	kodus, kad viskas būtų įskaitoma kaip reikia (tam taip pat ko gero reikės
	sutikrinti, ar korekcijų skaičiavimuose naudojamos failų versijos yra tinkamos)}.
	\item Pagal vadovo pastabas taisiau CMSDAS ataskaitą. Pagrindinės problemos buvo
	per daug bendrų, nieko konkretaus nepasakančių frazių vartojimas, per mažas
	asmeninės patirties perteikimas. Kad ką nors pamiršęs nepridaryčiau faktinių
	klaidų, užtrukau nemažai peržiūrėdamas iš naujo darytų užduočių skaidres,
	twiki puslapius ir pan.\ (visos užduotys gali būti rastos čia:
	\url{https://twiki.cern.ch/twiki/bin/viewauth/CMS/WorkBookExercisesCMSDataAnalysisSchool},
	paskaitų skaidres galima rasti čia: \url{https://indico.cern.ch/event/684249/timetable/#20180910})
\end{itemize}

\section{Spalio 11 d. -- CMSDAS ataskaitos taisymas}

\begin{itemize}
	\item Ant tier3 paleidau neinteraktyviai skaičiuoti su kitais procesais,
	kad patikrinčiau, ar problema tebėra.
	
	\textbf{UPDATE vakare:} paskaičiavau su procesais DYEE\_10to50, DYEE\_50to100,
	DYTauTau\_10to50, DYTauTau\_50to100, WJets, ttbar. Kol kas baigė skaičiuoti su
	Drell-Yan procesais, viskas veikė normaliai, WJets ir ttbar dar tebeskaičiuoja.
	Kita problema: kažkodėl po to, kai skaičiavimas baigiasi, HTCondor sistema vietoje
	to, kad mano paduotą darbą tiesiog užbaigtų (ir įvedus komandą condor\_q
	darbas būtų rodomas prie užbaigtų -- sekcijoje DONE), ji darbo vykdymą
	tiesiog pristabdo (rodo sekcijoje HELD). Galbūt taip gali būti dėl to,
	kad darbas nebūna atliekamas be jokių klaidų žinučių -- kažkodėl klaidų išvestyje
	gaunu tokį pranešimą:
	
	\ttt{WARNING: In non-interactive mode release checks e.g. deprecated releases,
	production architectures are disabled.\\
	**** Following environment variables are going to be unset.\\ SCONS\_LIB\_DIR}
	
	Nežinau, ką jis reiškia, bet skaičiavimui lyg ir netrukdo.
	\item Bandžiau dar kartą įsitikinti, ar tikrai eksperimentiniuose duomenyse
	dydis \ttt{nPileUp} visada yra lygus nuliui. Įsitikinau, kad taip, bet pastebėjau,
	kad taip pat duomenyse yra saugomas dydis \ttt{nVertices}, kuris eksperimentiniams
	duomenims jau nelygus nuliui. Tikrinau, kaip yra modeliuotuose duomenyse: ten taip
	pat yra dydis \ttt{nVertices}, bet šio dydžio pasiskirstymas nesutampa su
	\ttt{nPileUp} pasiskirstymu.
	\item Dar  porą kartų taisiau CMSDAS ataskaitą. Šį kartą pagrinde tai jau buvo rašybos, formulavimo
	arba minčių nuoseklumo klaidos.
	\item Taisiau duomenų naudojimo schemą ir pradėjau daryti duomenų saugojimo tier3 schemą.
\end{itemize}

\bibliography{\jobname}
\bibliographystyle{unsrt}

\end{document}
 
